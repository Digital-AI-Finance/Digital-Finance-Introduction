% template_summary.tex -- Shared preamble for module summaries
% Digital Finance Course -- Joerg Osterrieder
\documentclass[11pt,a4paper]{article}

% Packages
\usepackage[utf8]{inputenc}
\usepackage[T1]{fontenc}
\usepackage{lmodern}
\usepackage[margin=2.5cm]{geometry}
\usepackage{setspace}
\usepackage{parskip}
\usepackage{titlesec}
\usepackage{xcolor}
\usepackage{hyperref}
\usepackage{graphicx}
\usepackage{fancyhdr}
\usepackage{enumitem}
\usepackage{mdframed}

% Color scheme (matching course slides)
\definecolor{mlpurple}{RGB}{51,51,178}
\definecolor{mllavender}{RGB}{173,173,224}
\definecolor{mllavender2}{RGB}{193,193,232}
\definecolor{mllavender3}{RGB}{204,204,235}
\definecolor{mllavender4}{RGB}{214,214,239}
\definecolor{mlblue}{RGB}{0,102,204}
\definecolor{mlorange}{RGB}{255,127,14}
\definecolor{mlgreen}{RGB}{44,160,44}
\definecolor{mlred}{RGB}{214,39,40}
\definecolor{mlgray}{RGB}{127,127,127}

% Hyperlinks
\hypersetup{colorlinks=true, linkcolor=mlpurple, urlcolor=mlblue, citecolor=mlpurple}

% Section formatting (purple headings)
\titleformat{\section}{\Large\bfseries\color{mlpurple}}{}{0em}{}[\color{mllavender}\titlerule]
\titleformat{\subsection}{\large\bfseries\color{mlblue}}{}{0em}{}
\titleformat{\subsubsection}{\normalsize\bfseries\color{mlpurple}}{}{0em}{}

% Header/Footer
\pagestyle{fancy}
\fancyhf{}
\fancyhead[L]{\small\textcolor{mlgray}{Digital Finance -- Module Summary}}
\fancyhead[R]{\small\textcolor{mlgray}{\leftmark}}
\fancyfoot[C]{\small\textcolor{mlgray}{\thepage}}
\renewcommand{\headrulewidth}{0.4pt}
\renewcommand{\headrule}{\hbox to\headwidth{\color{mllavender}\leaders\hrule height \headrulewidth\hfill}}

% Custom environments
\newmdenv[
  linecolor=mlpurple,
  backgroundcolor=mllavender4,
  linewidth=2pt,
  leftline=true,
  rightline=false,
  topline=false,
  bottomline=false,
  innerleftmargin=10pt,
  innerrightmargin=10pt,
  innertopmargin=8pt,
  innerbottommargin=8pt
]{storybox}

\newmdenv[
  linecolor=mlorange,
  backgroundcolor=mlorange!5,
  linewidth=2pt,
  leftline=true,
  rightline=false,
  topline=false,
  bottomline=false,
  innerleftmargin=10pt,
  innerrightmargin=10pt,
  innertopmargin=8pt,
  innerbottommargin=8pt
]{insightbox}

\newmdenv[
  linecolor=mlgreen,
  backgroundcolor=mlgreen!5,
  linewidth=2pt,
  leftline=true,
  rightline=false,
  topline=false,
  bottomline=false,
  innerleftmargin=10pt,
  innerrightmargin=10pt,
  innertopmargin=8pt,
  innerbottommargin=8pt
]{handsonbox}

% Spacing
\setstretch{1.15}
\setlength{\parindent}{0pt}
\setlength{\parskip}{6pt}

% Custom commands
\newcommand{\modulenum}[1]{\textcolor{mlpurple}{\textbf{Module #1}}}
\newcommand{\topicref}[1]{\textcolor{mlblue}{\textit{#1}}}
\newcommand{\keyterm}[1]{\textbf{\textcolor{mlpurple}{#1}}}
\newcommand{\notebookref}[1]{\textcolor{mlgreen}{\texttt{#1}}}


\title{\textcolor{mlpurple}{\textbf{Module 2: Platform Finance}}\\[0.3cm]
{\Large How FinTech Reshapes Financial Services}}
\author{Joerg Osterrieder\\Digital Finance}
\date{}

\begin{document}

\maketitle
\thispagestyle{fancy}

% ============================================================================
% THE BIG QUESTION
% ============================================================================
\section{The Big Question}

How can a company that is not a bank offer you a bank account, a loan, and investment advice---all from a single app on your phone?

You have probably noticed that the boundary between ``tech company'' and ``financial institution'' has become almost impossible to draw. Ride-sharing apps offer driver bank accounts. E-commerce platforms extend merchant loans. Social media apps let you send money to friends. None of these companies started as banks, yet they now provide services that used to be the exclusive domain of licensed financial institutions.

This module pulls back the curtain on the infrastructure that makes this possible. You will discover that modern financial services rest on four interlocking pillars: a payment stack that moves money through hidden layers of intermediaries, an API economy that lets any authorised software tap into banking capabilities, data-driven algorithms that decide who gets credit and on what terms, and platform economics that determine which companies thrive and which quietly disappear. Understanding how these pillars fit together is the key to understanding modern finance itself.

% ============================================================================
% THE STORY
% ============================================================================
\section{The Story}

\begin{storybox}
Imagine a freelance graphic designer---call her Mira---who has just landed her first international client. She needs a way to receive payments from abroad, track her income, set aside money for taxes, and eventually qualify for a small business loan. A decade ago, she would have walked into a local bank branch, opened a business account, and hoped for the best. Today, she downloads an app.

Within minutes of signing up, Mira has a virtual debit card she can use immediately. The app automatically categorises her spending, flags upcoming tax obligations, and even offers her an overdraft facility. None of this feels unusual to Mira---it is simply how things work now. But behind that seamless experience, an extraordinary amount of machinery is humming away.

\textbf{The first payment arrives.} Mira's international client pays her through a card. That single transaction triggers a chain reaction across the entire \keyterm{four-layer payment stack}. At the top, the consumer interface---the client's banking app---initiates the transfer. One layer down, a payment processor translates the instruction into a message that card networks can understand. The card network routes the authorisation request from the client's issuing bank to Mira's acquiring bank. And at the very bottom, settlement systems---operated by central banks and clearing houses---eventually move the actual money. The whole authorisation takes seconds, but the money itself does not arrive for days. That gap between ``approved'' and ``settled'' is called \keyterm{float}, and it is one of the most profitable features of the entire payment system.

Mira barely notices the delay. What she does notice is the fee. A small percentage of every card payment disappears before it reaches her account. That percentage is \keyterm{interchange}---a fee that flows from her bank to the client's bank, passing through the card network along the way. Interchange exists because the client's bank bears the risk of fraud and non-payment. It funds the reward points in the client's wallet. Mira, like every merchant in the world, is the one who ultimately pays for those rewards---through slightly higher prices or slightly lower margins.

\textbf{Where does the app get its banking powers?} Mira's app is not a bank. It does not hold a banking licence. Instead, it uses \keyterm{Banking-as-a-Service}, or BaaS. Somewhere in the background, a licensed partner bank provides the actual account ledger, the deposit insurance, and the regulatory compliance. The app connects to that bank through \keyterm{APIs}---standardised contracts that let software systems talk to each other. Think of an API as a restaurant menu: it tells you exactly what you can order without requiring you to know how the kitchen works. The app also taps into \keyterm{open banking} regulations, which require banks to share customer data---with consent---through secure, token-based connections. Mira never hands over her password to anyone. Instead, her bank issues a temporary access token, like a concert wristband that grants limited access and can be revoked at any time.

\textbf{The loan decision.} After several months of steady income, Mira asks the app for a small business loan. Here is where \keyterm{data-driven finance} takes centre stage. A traditional bank would have pulled her credit report---and found almost nothing. Mira has no mortgage, no car loan, and only a short credit card history. She is what the industry calls a \keyterm{thin-file borrower}: someone with too little traditional data to generate a reliable score. But the app does not rely solely on traditional data. Because Mira has been using the app for months, it has a rich picture of her financial behaviour---her income consistency, her spending patterns, whether she pays recurring bills on time. These \keyterm{alternative data} signals feed into a machine learning model that discovers patterns a human loan officer would never spot. The model predicts Mira's creditworthiness and generates a score. The result: approved, at a competitive rate.

But there is a tension lurking beneath that approval. The same algorithms that include Mira can exclude others. If the training data reflects historical lending discrimination, the model learns to repeat it. Features like location can act as \keyterm{proxy variables}---appearing neutral while correlating with race or socioeconomic status. And because machine learning models are complex, they are harder to explain. When someone is denied credit, regulations in many jurisdictions require a clear explanation of why. Balancing prediction accuracy with \keyterm{explainability} and fairness is one of the defining challenges of modern finance.

\textbf{The business model question.} Mira loves the app. It is fast, elegant, and seems to offer everything for free. But ``free'' is never free. The app earns interchange revenue every time Mira uses her card. It earns interest on the deposits sitting in her account. It charges the partner bank a per-account fee. And it cross-sells premium features through a subscription tier. The real question is whether those revenue streams, taken together, exceed the cost of acquiring Mira as a customer in the first place. This is the domain of \keyterm{platform economics}: the study of \keyterm{network effects}, \keyterm{winner-take-most} dynamics, and the difference between real growth and venture-subsidised illusion. The strongest platforms spin a \keyterm{data flywheel}---more users generate more data, which improves the product, which attracts more users, which generates more data. The weakest ones burn through investor money offering below-cost services to users who will leave the moment prices rise.
\end{storybox}

% ============================================================================
% WHY THIS MATTERS
% ============================================================================
\section{Why This Matters}

The infrastructure described in Mira's story is not a niche curiosity. It is the foundation of modern financial services, and it touches your life whether you notice it or not.

If you pursue a career in banking, consulting, regulation, or technology, you will need to speak the language of payment stacks, API integrations, algorithmic decisioning, and platform dynamics. Product managers at technology companies make daily decisions about which payment methods to support and how to price them. Consultants advise banks on whether to build, buy, or partner their way into digital services. Regulators must decide how to supervise algorithms they cannot fully inspect. Even if your career takes you somewhere else entirely---law, healthcare, retail---you will encounter embedded finance: financial services woven directly into the platforms you use every day. Understanding how they work gives you a significant advantage over those who treat financial technology as a black box.

The stakes go beyond career relevance. The shift to data-driven, platform-based finance raises questions that affect everyone. Who should have access to credit, and on what terms? When an algorithm denies someone a loan, who is accountable---the developer who built the model, the lender who deployed it, or the regulator who approved it? If a handful of platforms dominate financial services, what happens to competition and consumer choice? If a FinTech's partner bank faces regulatory action, what happens to the customers who trusted that FinTech with their savings? These are not hypothetical questions. They are playing out right now, in regulatory debates, in courtrooms, and in the design decisions of the apps on your phone.

\begin{insightbox}
The line between ``technology company'' and ``financial institution'' has effectively dissolved. Understanding the infrastructure that erased it is no longer optional---it is foundational literacy for anyone who wants to participate meaningfully in the modern economy.
\end{insightbox}

% ============================================================================
% TOPIC DEEP DIVES
% ============================================================================

\subsection{Digital Payments}

Every time you tap a card, scan a code, or send money through an app, you trigger a multi-layered process that most people never see. The \keyterm{payment stack} has four layers, and each one extracts value from your transaction.

At the bottom sits \keyterm{Layer 1: Settlement Systems}---the infrastructure operated by central banks and clearing houses that actually moves money between financial institutions. Settlement is the final, irrevocable transfer of funds. It is slow (often taking days), but it is the bedrock on which everything else is built. Above it, \keyterm{Layer 2: Card Networks} route authorisation requests across borders, connecting the bank that issued your card (the \keyterm{issuing bank}) with the bank that serves the merchant (the \keyterm{acquiring bank}). Card networks exhibit powerful \keyterm{network effects}: the more cardholders they have, the more merchants want to accept them, and vice versa. \keyterm{Layer 3: Payment Processors} simplify the technical complexity of connecting to card networks. Before processors existed, accepting card payments required months of custom integration. Processors turned that into a simple API call. And at the top, \keyterm{Layer 4: Consumer Interfaces}---the apps, wallets, and checkout pages you interact with---compete fiercely for the right to own your customer relationship.

The economics of this stack centre on \keyterm{interchange}: the fee paid by the acquiring bank to the issuing bank on every card transaction. Interchange exists because the issuing bank bears the credit and fraud risk. It funds the reward programmes you enjoy. It is, in essence, a \keyterm{cross-subsidy}: merchants pay the fee, pass it into prices, and consumers pay indirectly whether they use a card or not.

Every payment method involves a tradeoff between speed, cost, and consumer protection. Card payments authorise quickly but settle slowly, and they offer chargebacks. Real-time payment rails---government-operated systems that settle in seconds at near-zero cost---offer speed and cheapness but no reversal. Wire transfers provide certainty but at high cost. There is no single best option; the right choice depends on context. Understanding this spectrum of tradeoffs is what separates an informed participant from a passive user of the system.

A newer development is reshaping this landscape: governments around the world are building \keyterm{real-time payment rails}---public infrastructure that settles bank-to-bank transfers in seconds at near-zero cost. When instant payments become free public infrastructure, the value shifts from moving money to the services built on top of money movement. This commoditises the basic rails and forces private payment companies to compete on experience, data, and ancillary services rather than on the transfer itself.

Cross-border payments remain the most expensive and slowest part of the payment infrastructure. Money must hop through a chain of \keyterm{correspondent banks}, each adding fees and delay. Each intermediary performs compliance checks and takes a cut, and currency conversion at every hop often occurs at unfavourable rates. The sender frequently does not know the total cost until the money arrives. This friction has made international payments a prime target for disruption---through local rail matching (where money never actually crosses borders), blockchain settlement (where a shared ledger removes intermediaries), and stablecoin transfers (where digital tokens move around the clock). No single approach dominates yet; the race to fix cross-border payments remains wide open.

\subsection{The API Economy and Banking-as-a-Service}

For decades, banks were closed systems. Their technology was proprietary, their data was siloed, and if you wanted to build something on top of banking infrastructure, you were out of luck. \keyterm{APIs}---Application Programming Interfaces---changed that by creating standardised contracts between software systems. An API defines what you can request and what you will receive, without requiring you to understand the internal workings of the system you are connecting to.

\keyterm{Open banking} took this further by making API access a regulatory mandate. In jurisdictions that have adopted it, banks are required to share customer data---with explicit consent---through secure, standardised APIs. Two key roles emerged: \keyterm{AISPs} (Account Information Service Providers), which have read-only access to your account data for purposes like budgeting and credit assessment, and \keyterm{PISPs} (Payment Initiation Service Providers), which can initiate payments from your account with your approval. The distinction matters because the risks differ: seeing your balance is far less dangerous than moving your money.

Security in this new world relies on \keyterm{token-based authentication} rather than password sharing. When you grant a third party access to your bank data, your bank issues a temporary, scoped access token. The third party never sees your password. You can revoke access at any time. This is fundamentally more secure than the old approach of ``screen scraping,'' where you handed your login credentials to a third party and hoped for the best.

\keyterm{Banking-as-a-Service} (BaaS) builds on top of APIs to let non-banks offer full financial products. A BaaS arrangement separates ``who builds the experience'' from ``who holds the licence.'' The FinTech builds the brand and the user interface---the part the customer sees and touches. Everything underneath---the banking charter, the compliance infrastructure, the account ledger, the deposit insurance---is rented from a licensed partner bank through a middleware platform. This dramatically lowers the barrier to entry for financial services, but it also creates dependency. A FinTech's fate is tied to its partner bank's regulatory standing.

\keyterm{Embedded finance} takes BaaS to its logical conclusion: financial services integrated directly into non-financial platforms at the moment of need. An e-commerce platform offers merchant loans based on sales data. A gig-work app provides instant payouts. When finance is embedded where you already are, the traditional bank becomes invisible. It still exists in the background, but the customer never interacts with it directly. In the API economy, value accrues to whoever owns the customer relationship---not necessarily to whoever holds the banking licence.

\subsection{Data-Driven Finance}

At the heart of every lending decision is a deceptively simple question: how does a lender decide whether to trust you with money they might never see again? Traditional \keyterm{credit scoring} answers this by condensing your financial history into a single number. That number---built primarily from your payment history, amounts owed, length of credit history, new credit applications, and credit mix---determines the cost of many major life decisions: your mortgage rate, your credit card limit, sometimes even whether you can rent an apartment.

The system works well for people who are already inside it. But millions of people are ``credit invisible''---not because they are risky, but because they lack traditional credit history. Young adults who have not had time to build a record, immigrants whose foreign credit history does not transfer, cash-economy users who pay for everything without formal credit---all fall into a catch-22 where you need credit to build a history and a history to get credit.

\keyterm{Alternative data} offers a way out. Bank transaction patterns, rent payments, utility bills, employment records---these everyday behaviours can reveal creditworthiness in people the traditional system ignores. A borrower who consistently pays rent on time and maintains stable income may be perfectly creditworthy, even without a single credit card to their name. \keyterm{Machine learning} models can discover these patterns at scale, processing hundreds of variables simultaneously to find signals that human analysts would never think to look for.

But better accuracy comes at a cost. Machine learning models are harder to explain. When a traditional model denies a loan, you can point to the specific factors and their weights. When a complex model denies a loan, the reasoning may be opaque even to its developers. This matters because in many jurisdictions, lenders are legally required to provide an \keyterm{adverse action notice}---a clear explanation of why an application was rejected. Techniques like \keyterm{SHAP values} help bridge this gap by assigning each input feature its ``fair share'' of a prediction, but the tension between accuracy and \keyterm{explainability} remains fundamental.

The deepest concern is \keyterm{algorithmic bias}. If past lending was discriminatory, the data reflects those patterns, and the model learns to repeat them. \keyterm{Proxy variables}---features that appear neutral but correlate with protected characteristics like race---can circumvent legal protections without anyone intending harm. Location can code for race through the legacy of historical residential segregation. School attended can correlate with socioeconomic status. Occupation type can correlate with gender. A model does not need to include a protected characteristic as an explicit variable to behave differently across groups; multiple proxies can combine to reconstruct the very attribute the law prohibits using. Addressing this requires active testing for \keyterm{disparate impact}---measuring whether a seemingly neutral policy disproportionately harms a protected group---and a commitment to fairness that goes beyond good intentions.

The data-driven approach also creates a powerful competitive dynamic: the \keyterm{data flywheel}. More loan applications improve risk models, better models enable more accurate pricing, better pricing attracts more borrowers, and more borrowers generate more data. This virtuous cycle compounds over time, creating advantages that new entrants cannot easily replicate. First-mover advantage in data is one of the strongest moats in modern finance.

\subsection{Platform Economics}

The most valuable companies in the modern economy do not make things---they connect people. A \keyterm{platform} creates value by facilitating exchanges between two or more interdependent groups, and this distinction from traditional ``pipeline'' businesses has profound implications for how FinTech companies compete, grow, and survive.

The engine of platform power is the \keyterm{network effect}: the phenomenon where a product becomes more valuable as more people use it. \keyterm{Direct network effects} occur on the same side---a peer payment app becomes more useful as more of your friends join. \keyterm{Indirect network effects} work across sides---more cardholders make a card network more attractive to merchants, which makes it more attractive to cardholders. \keyterm{Data network effects} emerge when more users generate better algorithms, as when a lending platform improves its risk models with every loan it makes.

Network effects create the conditions for \keyterm{winner-take-most} dynamics, where a single platform captures a dominant share of its market. But this only happens when three conditions align: strong network effects, high \keyterm{switching costs} (it is expensive or difficult to leave), and low \keyterm{multi-homing} (users find it impractical to use multiple platforms simultaneously). When multi-homing is easy, no single platform can lock users in, and competition persists. This is why card networks tend toward dominance while neobanks face relentless competition.

Every platform must solve the \keyterm{chicken-and-egg problem}: how do you launch a marketplace when each side needs the other to exist first? Successful platforms subsidise one side, offer standalone value before the network kicks in, or seed initial supply themselves. The strategy chosen at launch often shapes the company's economics for years.

The sustainability of a platform depends on its \keyterm{unit economics}. The \keyterm{lifetime value} (LTV) of each customer should be several times the \keyterm{customer acquisition cost} (CAC). The \keyterm{payback period}---how long until a customer ``pays back'' the cost of acquiring them---should be reasonable. And \keyterm{churn}---the rate at which customers leave---should be low enough that customers stay long enough to generate returns. When those numbers do not work, the company is surviving on venture subsidies rather than genuine demand. This is the logic of \keyterm{blitzscaling}: raise enormous capital, subsidise below-cost pricing, grow at all costs to reach critical mass, achieve network effects and lock-in, then raise prices once dominant. The strategy works brilliantly when the market truly tips and switching costs prevent departure. It fails catastrophically when multi-homing is easy, network effects are absent, or regulation prevents pricing power.

The critical test is simple: would customers stay at sustainable prices? If removing the subsidies would cause demand to collapse, the growth was artificial. The strongest platforms combine multiple moats---network effects, \keyterm{data flywheels}, switching costs, and regulatory positioning---to build compounding advantages that new entrants cannot easily replicate. Regulation itself can serve as a moat: once a company has invested in obtaining licences and building compliance infrastructure, competitors face the same expensive, time-consuming process to enter the market.

% ============================================================================
% HANDS-ON HIGHLIGHTS
% ============================================================================
\section{Hands-On Highlights}

\begin{handsonbox}
\textbf{\notebookref{NB02} -- Payment Transaction Analysis.} You work with a simulated payment dataset to compare payment methods across key dimensions: fees, settlement times, and network structure. You will build a payment network graph, identify hub participants, and visualise cost curves that reveal where different methods cross over. The goal is to draw business conclusions from data, not to write code from scratch.

\textbf{\notebookref{NB03} -- Open Banking API Simulation.} You build a simulated open banking environment from the ground up: a mock bank with customer accounts, API endpoints for reading account data and initiating payments, a token-based authentication flow, and a multi-bank aggregation dashboard. By the end, you will understand the mechanics of API-based banking through hands-on construction.

\textbf{\notebookref{NB04} -- Credit Scoring Model.} You walk through the entire credit scoring pipeline: exploring a dataset, engineering features from raw data, building both a simple traditional model and a more complex one, comparing their accuracy and interpretability, explaining individual predictions using feature importance, and testing for bias across demographic groups. This is where the accuracy--explainability tradeoff becomes tangible.
\end{handsonbox}

% ============================================================================
% KEY TAKEAWAYS
% ============================================================================
\section{Key Takeaways}

\begin{enumerate}[leftmargin=1.5em]
  \item \textbf{Payments flow through a four-layer stack}, and each layer---settlement systems, card networks, processors, consumer interfaces---extracts value from every transaction. Understanding this architecture reveals both where fees come from and where disruption is possible.

  \item \textbf{APIs unbundled banking.} Standardised interfaces let any authorised software access banking capabilities, enabling non-banks to offer financial products through Banking-as-a-Service without building infrastructure from scratch.

  \item \textbf{Open banking shifts power to consumers.} Your financial data belongs to you, not to your bank. Regulation-driven API access, secured by tokens rather than passwords, creates a level playing field between incumbents and innovators.

  \item \textbf{Data-driven scoring expands financial inclusion---but introduces new risks.} Alternative data and machine learning can bring credit-invisible people into the system, but algorithmic bias, proxy variables, and opacity demand constant vigilance.

  \item \textbf{Explainability is non-negotiable.} When algorithms make decisions that shape financial lives, consumers have a legal and ethical right to understand why---and to challenge those decisions.

  \item \textbf{Platform economics determine which FinTechs survive.} Network effects, switching costs, data flywheels, and unit economics separate sustainable businesses from venture-subsidised mirages. Always ask: would this work at real prices?

  \item \textbf{Value accrues to the customer interface.} In the API economy, whoever owns the customer relationship captures the most value---not necessarily whoever holds the banking licence or operates the underlying infrastructure.
\end{enumerate}

% ============================================================================
% LOOKING AHEAD
% ============================================================================
\section{Looking Ahead}

You now understand how money moves through layers of intermediaries, how APIs let any company offer financial services, how algorithms decide who gets credit, and how platform dynamics determine winners and losers. These are the mechanics of \textit{centralised} digital finance---systems that ultimately depend on trusted institutions, regulated intermediaries, and proprietary infrastructure.

Module 3 asks a radical question: what if you could remove those intermediaries entirely?

In \textbf{Module 3: Cryptographic Foundations and Blockchain}, you will explore the building blocks of a fundamentally different financial architecture. You will learn how cryptographic hash functions create tamper-proof records, how digital signatures prove ownership without revealing identity, and how distributed networks reach agreement without a central authority. These are not just abstract computer science concepts---they are the technical foundation of Bitcoin, Ethereum, and every decentralised financial system that followed.

The connection to what you have learned here is direct. The payment stack you studied in this module has four layers, each extracting fees. Blockchain asks: what if you could collapse those layers? The BaaS model depends on licensed partner banks. Decentralised finance asks: what if the licence itself became unnecessary? The data flywheel rewards centralisation. Blockchain-based systems explore whether decentralised networks can compete.

Whether decentralised alternatives will complement or replace the platform finance infrastructure you have just studied is one of the defining questions of digital finance. Module 3 gives you the tools to form your own answer.

\end{document}
