% template_summary.tex -- Shared preamble for module summaries
% Digital Finance Course -- Joerg Osterrieder
\documentclass[11pt,a4paper]{article}

% Packages
\usepackage[utf8]{inputenc}
\usepackage[T1]{fontenc}
\usepackage{lmodern}
\usepackage[margin=2.5cm]{geometry}
\usepackage{setspace}
\usepackage{parskip}
\usepackage{titlesec}
\usepackage{xcolor}
\usepackage{hyperref}
\usepackage{graphicx}
\usepackage{fancyhdr}
\usepackage{enumitem}
\usepackage{mdframed}

% Color scheme (matching course slides)
\definecolor{mlpurple}{RGB}{51,51,178}
\definecolor{mllavender}{RGB}{173,173,224}
\definecolor{mllavender2}{RGB}{193,193,232}
\definecolor{mllavender3}{RGB}{204,204,235}
\definecolor{mllavender4}{RGB}{214,214,239}
\definecolor{mlblue}{RGB}{0,102,204}
\definecolor{mlorange}{RGB}{255,127,14}
\definecolor{mlgreen}{RGB}{44,160,44}
\definecolor{mlred}{RGB}{214,39,40}
\definecolor{mlgray}{RGB}{127,127,127}

% Hyperlinks
\hypersetup{colorlinks=true, linkcolor=mlpurple, urlcolor=mlblue, citecolor=mlpurple}

% Section formatting (purple headings)
\titleformat{\section}{\Large\bfseries\color{mlpurple}}{}{0em}{}[\color{mllavender}\titlerule]
\titleformat{\subsection}{\large\bfseries\color{mlblue}}{}{0em}{}
\titleformat{\subsubsection}{\normalsize\bfseries\color{mlpurple}}{}{0em}{}

% Header/Footer
\pagestyle{fancy}
\fancyhf{}
\fancyhead[L]{\small\textcolor{mlgray}{Digital Finance -- Module Summary}}
\fancyhead[R]{\small\textcolor{mlgray}{\leftmark}}
\fancyfoot[C]{\small\textcolor{mlgray}{\thepage}}
\renewcommand{\headrulewidth}{0.4pt}
\renewcommand{\headrule}{\hbox to\headwidth{\color{mllavender}\leaders\hrule height \headrulewidth\hfill}}

% Custom environments
\newmdenv[
  linecolor=mlpurple,
  backgroundcolor=mllavender4,
  linewidth=2pt,
  leftline=true,
  rightline=false,
  topline=false,
  bottomline=false,
  innerleftmargin=10pt,
  innerrightmargin=10pt,
  innertopmargin=8pt,
  innerbottommargin=8pt
]{storybox}

\newmdenv[
  linecolor=mlorange,
  backgroundcolor=mlorange!5,
  linewidth=2pt,
  leftline=true,
  rightline=false,
  topline=false,
  bottomline=false,
  innerleftmargin=10pt,
  innerrightmargin=10pt,
  innertopmargin=8pt,
  innerbottommargin=8pt
]{insightbox}

\newmdenv[
  linecolor=mlgreen,
  backgroundcolor=mlgreen!5,
  linewidth=2pt,
  leftline=true,
  rightline=false,
  topline=false,
  bottomline=false,
  innerleftmargin=10pt,
  innerrightmargin=10pt,
  innertopmargin=8pt,
  innerbottommargin=8pt
]{handsonbox}

% Spacing
\setstretch{1.15}
\setlength{\parindent}{0pt}
\setlength{\parskip}{6pt}

% Custom commands
\newcommand{\modulenum}[1]{\textcolor{mlpurple}{\textbf{Module #1}}}
\newcommand{\topicref}[1]{\textcolor{mlblue}{\textit{#1}}}
\newcommand{\keyterm}[1]{\textbf{\textcolor{mlpurple}{#1}}}
\newcommand{\notebookref}[1]{\textcolor{mlgreen}{\texttt{#1}}}


\title{%
  \textcolor{mlpurple}{\textbf{Module 1: Why Digital Finance?}}\\[6pt]
  \Large\textcolor{mlblue}{From Friction to Innovation}
}
\author{Joerg Osterrieder\\Digital Finance}
\date{}

\begin{document}

\maketitle
\thispagestyle{fancy}

% ===========================================================================
% SECTION: THE BIG QUESTION
% ===========================================================================
\section{The Big Question}

You want to send money to a friend in another country. It should be as easy as sending a text message. You open your banking app, type in the amount, and hit send. Then you wait. And wait. The money vanishes from your account almost immediately, but your friend will not see it for days. Along the way, mysterious fees nibble at the total. By the time the transfer arrives, a noticeable chunk has disappeared into the hands of institutions you have never met.

Why? You can send a photo across the globe in under a second, for free. You can video-call someone on the other side of the planet without thinking twice about the cost. Information moves at the speed of light. So why does money still crawl?

This is not a minor inconvenience. For hundreds of millions of people around the world---migrant workers sending earnings back to their families, small businesses paying international suppliers, freelancers collecting payments from overseas clients---these delays and fees are not just annoying. They are expensive, exclusionary, and sometimes devastating. A significant fraction of every remittance is lost to fees, money that could have fed a family or paid school tuition.

The answer to ``why is this so hard?'' turns out to be far more interesting than you might expect. It leads you down a path that questions everything you thought you knew about money, trust, and the architecture of the financial system itself. By the end of this module, you will understand not just the problem, but the two radically different philosophies that have emerged to solve it---and why both of them matter for your future.

% ===========================================================================
% SECTION: THE STORY
% ===========================================================================
\section{The Story}

\begin{storybox}
Imagine you are a university student. Your close friend has moved abroad for a semester exchange, and her birthday is coming up. You decide to send her some money as a gift---simple, personal, done. You open your banking app and initiate a transfer. The interface looks modern, sleek, instant. But behind that polished screen, something very different is happening.
\end{storybox}

Your money does not travel in a straight line. It hops. First, it goes from your bank to an intermediary---a \keyterm{correspondent bank} that has a relationship with a bank in your friend's country. That correspondent bank passes it along to another intermediary, which finally routes it to her bank. Each hop takes time. Each hop costs money. Each hop introduces a tiny chance that something goes wrong: a compliance hold, a currency conversion error, a batch-processing delay because the transfer happened after business hours on a Friday.

You start to wonder: who are all these intermediaries, and why do they exist?

\begin{storybox}
To answer that question, you have to go deeper---all the way back to the most fundamental question in finance. What \emph{is} money? You probably think of money as a thing: coins in your pocket, numbers on a screen, a balance in your bank account. But money is not a thing at all. Money is an agreement. It is a system of record-keeping---a \keyterm{ledger}---backed by trust. When your bank says you have a certain balance, what it really means is that a database somewhere records that the world owes you that amount. Money is information, and it always has been. Debt and credit systems, scholars have shown, predate physical currency by thousands of years.
\end{storybox}

This realization changes everything. If money is just information, then moving money should be as easy as moving any other kind of information. But here is the catch: information can be copied. If you email someone a photo, you still have the photo. If money worked the same way---if you could copy a digital dollar and spend it twice---the entire concept of value would collapse. This is the \keyterm{double-spending problem}, and it is the central puzzle of digital money. Physical cash solves it automatically: if you hand someone a banknote, you no longer have it. The laws of physics enforce scarcity. But digital files have no such constraint.

\begin{storybox}
For centuries, the solution was straightforward: put someone in charge. A bank, a government, a trusted authority that maintains a single ledger and validates every transaction. When you transfer money, the bank checks your balance, deducts the amount, and credits the recipient. Double-spending prevented. Problem solved. But this solution carries a heavy price tag. The bank charges fees. The bank operates on its own schedule. The bank decides who gets to participate and who does not. The bank can freeze your account, block your transactions, and share your data. The bank can even fail entirely, as many did during the financial crisis. You are trusting an institution with your financial life, and that trust is not free.
\end{storybox}

This brings you to the six \keyterm{pain points} that define the traditional financial system. \emph{Speed}: money moves slowly because it passes through multiple intermediaries running legacy systems designed decades ago, processing transactions in batches rather than in real time. \emph{Cost}: every intermediary takes a cut, and those cuts add up---especially for cross-border transfers and small amounts. \emph{Access}: billions of people worldwide lack a bank account because they cannot meet the identification, documentation, or minimum-balance requirements. \emph{Transparency}: the system is opaque---hidden fees, complex pricing structures, and information asymmetries that consistently favor institutions over individuals. \emph{Fragmentation}: different payment networks, different standards, different regulatory regimes that do not talk to each other. \emph{Availability}: traditional finance operates on business hours, in specific time zones, with weekends and holidays off---while the global economy never sleeps.

\begin{storybox}
Standing at this crossroads, you see two paths forward. The first says: the existing system works, it just needs to be better. Make the interfaces smoother, the processes faster, the fees lower. Use technology---mobile apps, APIs, machine learning, cloud computing---to improve the experience without tearing down the infrastructure. This is the \keyterm{FinTech} philosophy: better user experience on existing rails.

The second path says something bolder: the infrastructure itself is the problem. We do not need to make the old system faster. We need to build a new one from scratch---one where trust comes not from institutions but from mathematics, where ledgers are distributed rather than centralized, where access requires nothing more than an internet connection. This is the \keyterm{Crypto/DeFi} philosophy: new rails, new rules.
\end{storybox}

Neither path is universally right. A FinTech neobank might be the perfect solution for someone who wants a better mobile banking experience with consumer protections and customer support. A decentralized lending protocol might be the only option for someone in a country with an unstable currency and no access to a bank. The two philosophies address the same frictions from fundamentally different directions: one works \emph{within} the system, the other builds \emph{around} it. And increasingly, the boundary between them is blurring, as FinTech companies add crypto features and crypto projects improve their user experience toward mainstream standards.

\begin{storybox}
Finally, you step back and see the full landscape. Digital finance is not one thing---it is an interconnected ecosystem organized into six major sectors: payments, lending, trading, investing, insurance, and banking infrastructure. Beneath all of them lies a shared foundation of technology: blockchain networks, APIs, data systems, and identity infrastructure. Every innovation you will encounter in this course fits somewhere on this map. Understanding the map is what turns a bewildering collection of buzzwords into a coherent picture of where finance is heading.

Your birthday transfer? It is a tiny thread in an enormous tapestry. But by pulling on that thread, you have unraveled the whole story: what money is, why it is hard to move, who profits from the friction, and how two competing visions are racing to fix it.
\end{storybox}

% ===========================================================================
% SECTION: WHY THIS MATTERS
% ===========================================================================
\section{Why This Matters}

You might be wondering why any of this matters to you personally. The answer is that digital finance is reshaping every industry, every career path, and every economy on the planet---and the transformation is accelerating.

If you work in banking, insurance, or asset management, the products you sell and the infrastructure you rely on are being rebuilt in real time. If you work in technology, the fastest-growing opportunities are in financial services. If you are an entrepreneur, the barriers to offering financial products---payments, lending, investing---have never been lower, thanks to APIs and embedded finance. If you work in policy or regulation, you will spend your career navigating the tension between innovation and consumer protection. And even if finance is not your field at all, your personal financial life---how you save, invest, borrow, insure, and transact---is being transformed by the forces described in this module.

Understanding the ``why'' behind digital finance gives you a framework for evaluating every new product, platform, and protocol you encounter. Instead of asking ``is this cool?'' you can ask the more powerful questions: What friction does this address? Who benefits? What are the trade-offs? Is this improving the existing system or building a parallel one? These are the questions that separate informed participants from passive consumers.

\begin{insightbox}
\textbf{Core Insight:} Every FinTech and Crypto/DeFi innovation targets a specific friction in the traditional financial system. Understanding those frictions---and who bears their costs---is the single most important skill for navigating the future of finance.
\end{insightbox}

% ===========================================================================
% SUBSECTION: WHAT IS MONEY, REALLY?
% ===========================================================================
\subsection{What Is Money, Really?}

The first topic in this module dissolves your assumptions about money and rebuilds the concept from scratch. Money is not gold, paper, or numbers on a screen. Money is a \keyterm{social technology}---a shared system of trust and record-keeping that allows strangers to exchange value.

Money performs three functions. As a \keyterm{medium of exchange}, it eliminates the need for barter by providing something everyone accepts. As a \keyterm{unit of account}, it gives us a common yardstick for measuring the value of different goods and services. As a \keyterm{store of value}, it allows you to save purchasing power for the future. Anything that serves all three functions can act as money---from shells and gold to digital tokens---as long as enough people believe it will be accepted by others.

The history of money reveals a consistent pattern: each stage becomes more abstract, more scalable, and more dependent on trust. Barter requires that both parties want exactly what the other has. Commodity money (gold, silver) solves this but is heavy and hard to divide. Paper currency is lighter and more flexible but depends on the promise of whoever issued it. Digital money---the numbers in your bank account---is the most abstract of all: it exists purely as entries in a \keyterm{ledger}.

A ledger is simply a record of who owes what to whom. It is the oldest financial technology in existence, predating physical currency by millennia. Every financial system, from ancient Mesopotamian clay tablets to modern central banking, is fundamentally a ledger system. The critical question is always the same: \emph{who maintains the ledger, and why should we trust them?}

This is where the \keyterm{double-spending problem} enters. Physical objects can only be in one place at a time, so handing over a coin automatically prevents you from spending it twice. But digital files can be copied for free, infinitely. If money is just a file, what stops someone from duplicating it and spending the same money over and over? Before blockchain, the only known solution was to appoint a trusted central authority---a bank---to maintain a single authoritative ledger. The bank checks every transaction, ensures the sender has sufficient funds, and updates the records. Double-spending prevented.

But the costs of this arrangement are significant: fees, delays, exclusion, censorship risk, and \keyterm{counterparty risk}---the possibility that the trusted party itself might fail. The global financial crisis demonstrated that this risk is not hypothetical. The quest for an alternative---a way to prevent double-spending without trusting any single party---is what ultimately led to blockchain technology.

\begin{insightbox}
\textbf{Key Insight:} Money is not a thing. It is a system of trust. Understanding this changes how you evaluate every financial innovation: the real question is always ``how does this system establish and maintain trust?''
\end{insightbox}

% ===========================================================================
% SUBSECTION: FINANCIAL SYSTEM PAIN POINTS
% ===========================================================================
\subsection{Financial System Pain Points}

The second topic maps the six core frictions that plague the traditional financial system. These frictions are not random flaws---they are structural consequences of how the system was designed, and each one represents a massive opportunity for innovation.

\textbf{Slow Settlement.} When you buy a stock, the trade executes instantly, but the actual transfer of money and securities takes an additional business day (known as T+1 settlement). International wire transfers take even longer. The reason is a combination of batch processing, timezone differences, multiple intermediaries, and legacy technology. During the settlement window, capital is locked, counterparty risk is elevated, and the entire system is vulnerable to cascading failures if a major participant defaults.

\textbf{High Fees.} Cross-border transfers are the starkest example. The global average cost of sending a remittance remains well above the target set by international development goals. For the hundreds of millions of migrant workers who send earnings home, these fees represent a direct tax on poverty. The fees come from multiple sources: originating bank charges, correspondent bank fees, messaging network costs, receiving bank fees, and hidden exchange-rate markups.

\textbf{Financial Exclusion.} Over a billion adults worldwide lack access to a basic bank account. The barriers are numerous: lack of government-issued identification, geographic distance from bank branches, minimum-balance requirements, and documentation demands. The consequences cascade: without a bank account, you cannot save securely, access credit, obtain insurance, or receive government payments efficiently. Those who need financial services the most have the least access to them.

\textbf{Opacity.} The financial system is riddled with \keyterm{information asymmetry}---situations where institutions know far more than their customers. Hidden fees in investment products, opaque pricing in insurance, payment-for-order-flow arrangements in stock trading---these practices consistently transfer value from retail customers to institutions.

\textbf{Fragmentation.} Different payment networks use different standards, different message formats, and different settlement cycles. Moving money between systems requires intermediaries, each adding cost and delay. The incompatibility between domestic and international payment rails is a particularly acute example.

\textbf{Limited Hours.} Traditional finance operates on business-hour schedules, in local time zones, with weekends and holidays off. This is fundamentally misaligned with a global economy that operates continuously. A transfer initiated on a Friday evening may not settle until the following Tuesday.

A critical observation runs through all six frictions: their costs are \keyterm{regressive}. They take a larger proportional bite from those who have less. A flat wire-transfer fee that is negligible for a large corporation is devastating for a low-income worker sending a small amount home. This is not just an economic inefficiency---it is a social justice issue.

\begin{insightbox}
\textbf{Key Insight:} Financial frictions are not just inconveniences---they are regressive costs that disproportionately harm the people who can least afford them. This is why reducing friction through technology can be inherently progressive.
\end{insightbox}

% ===========================================================================
% SUBSECTION: FINTECH VS. CRYPTO/DEFI
% ===========================================================================
\subsection{FinTech vs.\ Crypto/DeFi}

The third topic introduces the fundamental philosophical fork that defines the digital finance landscape. Both FinTech and Crypto/DeFi target the same frictions, but they approach them from opposite directions.

\textbf{The FinTech Philosophy: ``Improve the System.''} FinTech companies believe the existing financial infrastructure works---it just needs better interfaces, smarter processes, and more efficient technology. They build \emph{on top of} existing rails: bank networks, card networks, payment processors, regulatory frameworks. A FinTech neobank still partners with a licensed bank to hold deposits. A FinTech payment app still settles through the same clearing systems. The innovation is in the experience layer: lower fees, faster approvals, mobile-first design, and data-driven personalization. The trust model remains \keyterm{institutional}: you trust the company, which is backed by regulators, deposit insurance, and legal recourse.

\textbf{The Crypto/DeFi Philosophy: ``Replace the System.''} The Crypto/DeFi approach argues that the infrastructure itself is the bottleneck. Instead of layering better technology on top of legacy systems, it builds entirely new infrastructure from scratch. The trust model is \keyterm{cryptographic}: you trust the code, the math, and the consensus mechanism---not any institution. Smart contracts execute automatically when conditions are met. Access is permissionless---anyone with an internet connection can participate. Transactions are transparent and recorded on a public ledger. There is no central authority that can freeze your account, block your transfer, or decide you are not allowed to participate.

The difference is not just technical---it is about values. FinTech prioritizes convenience, consumer protection, and regulatory compliance. Crypto/DeFi prioritizes autonomy, censorship resistance, and permissionless access. FinTech says: ``We have got your back.'' DeFi says: ``You are on your own---but no one can stop you.''

This creates a fundamental trade-off. In the FinTech world, if something goes wrong---a fraudulent transaction, a company collapse, an error---there are safety nets: chargebacks, deposit insurance, legal recourse, customer support. In the DeFi world, transactions are irreversible, lost private keys mean lost funds forever, and there is often no one to call for help.

Between these two poles lies a growing \keyterm{hybrid zone}. Centralized crypto exchanges require identity verification but offer crypto trading. FinTech companies are adding crypto features to their apps. The boundary is blurring, but the underlying philosophical tension remains. Understanding where any given innovation sits on this spectrum---and what trade-offs it makes---is one of the most important analytical skills you can develop in digital finance.

\begin{insightbox}
\textbf{Key Insight:} The choice between FinTech and Crypto/DeFi is not about which technology is better. It is about which trade-offs you prefer: protection with less freedom, or freedom with less protection. Neither is universally superior---the right choice depends on the specific use case, the user's circumstances, and the regulatory environment.
\end{insightbox}

% ===========================================================================
% SUBSECTION: THE DIGITAL FINANCE LANDSCAPE
% ===========================================================================
\subsection{The Digital Finance Landscape}

The fourth topic provides the map. Digital finance is organized into six interconnected sectors, each existing in both FinTech and Crypto/DeFi forms, all built on a shared infrastructure layer.

\textbf{Payments} is the foundation. This sector covers everything from splitting a dinner bill with a peer-to-peer app to processing merchant transactions to sending cross-border remittances. It is the entry point for most users and the highest-volume sector by far. FinTech approaches include mobile payment apps and low-cost transfer services. Crypto approaches include stablecoins and blockchain-based settlement.

\textbf{Lending} encompasses consumer loans, small-business credit, peer-to-peer lending, and the rapidly growing buy-now-pay-later category. FinTech lenders use alternative data and machine learning to make faster, more inclusive credit decisions. DeFi lending protocols use smart contracts and over-collateralization to enable instant, permissionless borrowing.

\textbf{Trading and Exchanges} covers stock trading, cryptocurrency exchanges, derivatives, and tokenized assets. The key innovation on the FinTech side has been the democratization of access through commission-free trading and fractional shares. On the DeFi side, \keyterm{decentralized exchanges} use \keyterm{automated market makers}---liquidity pools governed by mathematical formulas---to enable trading without order books or central operators.

\textbf{Investing and Wealth Management} includes robo-advisors that brought professional portfolio management to anyone, micro-investing apps that let you start with trivial amounts, and DeFi yield strategies that automate returns across multiple protocols.

\textbf{Insurance} is being reinvented on both sides: InsurTech companies use AI for instant claims processing and data-driven pricing, while DeFi protocols offer decentralized coverage against smart-contract failures and protocol risks.

\textbf{Banking Infrastructure} is the plumbing beneath everything else. This includes neobanks (digital-only banks with no branches), Banking-as-a-Service platforms (which let any company offer financial products through APIs), and open banking standards that mandate data sharing between institutions.

Beneath all six sectors lies the \keyterm{infrastructure layer}: blockchain networks that provide settlement and execution, APIs that connect systems, data services (including oracles that bridge on-chain and off-chain information), and identity systems ranging from traditional KYC to decentralized identifiers.

Finally, several \keyterm{emerging categories} are reshaping the boundaries of the map. Tokenization is turning illiquid assets---real estate, art, carbon credits---into tradable digital tokens. Decentralized Autonomous Organizations (DAOs) are experimenting with code-based governance. Central Bank Digital Currencies (CBDCs) represent governments entering the digital money space directly. And artificial intelligence is being woven into every sector, from credit scoring to fraud detection to automated advising.

The power of this landscape framework is that it lets you place any innovation in context. When you encounter a new product or protocol, you can immediately ask: which sector does it serve, which philosophy does it follow, which friction does it target, and what infrastructure does it depend on?

\begin{insightbox}
\textbf{Key Insight:} Digital finance is not a collection of isolated technologies. It is an interconnected ecosystem where sectors depend on each other and share common infrastructure. Understanding the map---not just individual innovations---is what gives you strategic vision.
\end{insightbox}

% ===========================================================================
% SECTION: HANDS-ON HIGHLIGHTS
% ===========================================================================
\section{Hands-On Highlights}

\begin{handsonbox}
In \notebookref{NB01} (Money and Ledgers), you get to experience the core concepts of this module through code. You do not need any programming background---the notebook is designed so you can read along and understand what is happening at each step.

You will:
\begin{itemize}[nosep]
  \item \textbf{Build a simple ledger} from scratch---creating accounts, recording transactions, and tracking balances---to see how every financial system works at its core.
  \item \textbf{Execute a double-spending attack} on an unprotected ledger, watching in real time as money is created out of thin air when there is no validation.
  \item \textbf{Implement a trusted bank} that validates balances before every transfer, and see the attack fail---demonstrating exactly how centralized trust solves the problem.
  \item \textbf{Visualize the cost of trust} by comparing the fees, delays, and trade-offs across different transfer methods, from traditional wire transfers to stablecoins.
  \item \textbf{Compare money systems} across multiple dimensions---speed, privacy, accessibility, security---to see that no single system is best at everything.
\end{itemize}

The notebook transforms the abstract concepts from the slides into concrete, interactive demonstrations. Seeing a double-spending attack succeed and then fail is far more powerful than reading about it.
\end{handsonbox}

% ===========================================================================
% SECTION: KEY TAKEAWAYS
% ===========================================================================
\section{Key Takeaways}

\begin{enumerate}[leftmargin=*, itemsep=6pt]

  \item \textbf{Money is a trust system, not a physical object.} It is information recorded in ledgers, backed by collective belief. Understanding this is the foundation for everything else in digital finance.

  \item \textbf{The double-spending problem is the central challenge of digital money.} Physical objects cannot be in two places at once, but digital files can be copied infinitely. Every digital money system must solve this problem.

  \item \textbf{Traditional finance solves double-spending through trusted intermediaries---at a significant cost.} Banks prevent fraud but introduce fees, delays, exclusion, censorship risk, opacity, and counterparty risk.

  \item \textbf{Financial frictions are regressive.} They disproportionately harm those with less money, less access, and less information. Reducing friction is not just an efficiency gain---it is a step toward equity.

  \item \textbf{Two philosophies compete to fix the system.} FinTech improves the existing infrastructure with better technology and user experience. Crypto/DeFi builds entirely new infrastructure based on cryptographic trust and permissionless access. Neither is universally superior.

  \item \textbf{The digital finance landscape is an interconnected ecosystem.} Six sectors (payments, lending, trading, investing, insurance, banking) share a common infrastructure layer and are increasingly converging across the FinTech--DeFi divide.

  \item \textbf{The analytical skill that matters most is asking the right questions.} For any innovation: What friction does it address? Who benefits? What are the trade-offs? Is it improving the existing system or building a new one?

\end{enumerate}

% ===========================================================================
% SECTION: LOOKING AHEAD
% ===========================================================================
\section{Looking Ahead}

Module 1 established the \emph{why} of digital finance: what money is, what is broken, and two philosophies for fixing it. Module 2 goes deep on the \emph{how}---specifically, how the FinTech side of the landscape actually works.

You will explore digital payments in detail: the rails beneath the apps, the difference between what looks instant on your screen and what actually happens behind the scenes, and the global payment systems that move trillions daily. You will learn about the \keyterm{API economy} and Banking-as-a-Service---the infrastructure that allows any company to embed financial products into its platform. You will see how data is transforming credit decisions, fraud detection, and personalized financial services. And you will encounter \keyterm{platform economics}: the network effects, multi-sided markets, and winner-take-most dynamics that explain why a handful of platforms dominate digital finance.

If Module 1 gave you the map, Module 2 hands you the first set of keys---to payments, platforms, and the data-driven economy that powers modern FinTech.

\end{document}
