% Platform Economics: Pre-Class Intro
% Neutral, fact-based overview with comics and charts
\documentclass[11pt,aspectratio=169]{beamer}
\usetheme{Madrid}

% ======================= PACKAGES =======================
\usepackage{graphicx}
\usepackage{booktabs}
\usepackage{adjustbox}
\usepackage{multicol}
\usepackage{amsmath}
\usepackage{amssymb}
\usepackage{tikz}
\usetikzlibrary{arrows,shapes,positioning,shadows,trees}
\usepackage{listings}
\usepackage{xcolor}

% ======================= COLOR DEFINITIONS =======================
% Primary color scheme: Blue/Teal for Digital Finance
\definecolor{dfblue}{RGB}{0,102,204}
\definecolor{dfteal}{RGB}{0,153,153}
\definecolor{dfcyan}{RGB}{51,187,204}
\definecolor{dflightblue}{RGB}{153,204,255}
\definecolor{dflightblue2}{RGB}{173,214,255}
\definecolor{dflightblue3}{RGB}{193,224,255}
\definecolor{dflightblue4}{RGB}{213,234,255}

% Accent colors for finance applications
\definecolor{dfgreen}{RGB}{44, 160, 44}
\definecolor{dfred}{RGB}{214, 39, 40}
\definecolor{dforange}{RGB}{255, 127, 14}
\definecolor{dfgray}{RGB}{127, 127, 127}

% Utility colors
\definecolor{lightgray}{RGB}{240, 240, 240}
\definecolor{midgray}{RGB}{180, 180, 180}
\definecolor{codebg}{RGB}{245, 245, 245}

% ======================= THEME CUSTOMIZATION =======================
% Apply Digital Finance color scheme to Madrid theme
\setbeamercolor{palette primary}{bg=dflightblue3,fg=dfblue}
\setbeamercolor{palette secondary}{bg=dflightblue2,fg=dfblue}
\setbeamercolor{palette tertiary}{bg=dfteal,fg=white}
\setbeamercolor{palette quaternary}{bg=dfblue,fg=white}

\setbeamercolor{structure}{fg=dfblue}
\setbeamercolor{section in toc}{fg=dfblue}
\setbeamercolor{subsection in toc}{fg=dfteal}
\setbeamercolor{title}{fg=dfblue}
\setbeamercolor{frametitle}{fg=dfblue,bg=dflightblue3}
\setbeamercolor{block title}{bg=dflightblue2,fg=dfblue}
\setbeamercolor{block body}{bg=dflightblue4,fg=black}

% Remove navigation symbols for cleaner look
\setbeamertemplate{navigation symbols}{}

% Clean itemize/enumerate
\setbeamertemplate{itemize items}[circle]
\setbeamertemplate{enumerate items}[default]

% Margins for readability
\setbeamersize{text margin left=8mm,text margin right=8mm}

% ======================= LISTINGS CONFIGURATION =======================
% Python code style
\lstdefinestyle{pythonstyle}{
    language=Python,
    basicstyle=\ttfamily\footnotesize,
    keywordstyle=\color{dfblue}\bfseries,
    stringstyle=\color{dforange},
    commentstyle=\color{dfgray}\itshape,
    numberstyle=\tiny\color{dfgray},
    numbers=left,
    numbersep=5pt,
    backgroundcolor=\color{codebg},
    showspaces=false,
    showstringspaces=false,
    showtabs=false,
    frame=single,
    rulecolor=\color{midgray},
    tabsize=4,
    captionpos=b,
    breaklines=true,
    breakatwhitespace=false,
    escapeinside={(*@}{@*)},
    xleftmargin=10pt,
    xrightmargin=10pt
}

% Solidity code style
\lstdefinestyle{soliditystyle}{
    language=Java, % closest approximation
    basicstyle=\ttfamily\footnotesize,
    keywordstyle=\color{dfteal}\bfseries,
    stringstyle=\color{dforange},
    commentstyle=\color{dfgray}\itshape,
    numberstyle=\tiny\color{dfgray},
    numbers=left,
    numbersep=5pt,
    backgroundcolor=\color{codebg},
    showspaces=false,
    showstringspaces=false,
    showtabs=false,
    frame=single,
    rulecolor=\color{midgray},
    tabsize=2,
    captionpos=b,
    breaklines=true,
    breakatwhitespace=false,
    escapeinside={(*@}{@*)},
    xleftmargin=10pt,
    xrightmargin=10pt,
    morekeywords={pragma, contract, function, returns, public, private, view, pure, payable, address, uint256, mapping, event, modifier}
}

% Inline code command
\newcommand{\code}[1]{\texttt{\color{dfblue}#1}}

% ======================= CUSTOM COMMANDS =======================
% Bottom annotation (Madrid-style)
\newcommand{\bottomnote}[1]{%
\vfill
\vspace{-2mm}
\textcolor{dflightblue2}{\rule{\textwidth}{0.4pt}}
\vspace{1mm}
\footnotesize
\textbf{#1}
}

% Compact list spacing
\newcommand{\compactlist}{%
\setlength{\itemsep}{0pt}%
\setlength{\parskip}{0pt}%
\setlength{\parsep}{0pt}%
}

% Chart placeholder
\newcommand{\chartplaceholder}[2][5cm]{%
\begin{center}
\begin{adjustbox}{max width=0.95\textwidth, max height=#1}
\framebox[\textwidth][c]{%
\rule{0pt}{#1}%
\textcolor{midgray}{[#2]}%
}
\end{adjustbox}
\end{center}
}

% ======================= FINANCE NOTATION MACROS =======================
% Probability and statistics
\newcommand{\E}{\mathbb{E}} % Expected value
\newcommand{\Var}{\mathrm{Var}} % Variance
\newcommand{\Cov}{\mathrm{Cov}} % Covariance
\newcommand{\Prob}{\mathbb{P}} % Probability

% Distributions
\newcommand{\Normal}{\mathcal{N}} % Normal distribution
\newcommand{\Uniform}{\mathcal{U}} % Uniform distribution

% Returns and prices
\newcommand{\Ret}{R} % Return
\newcommand{\LogRet}{r} % Log return
\newcommand{\Price}{S} % Price/Stock price
\newcommand{\Strike}{K} % Strike price

% Options and derivatives
\newcommand{\CallPrice}{C} % Call option price
\newcommand{\PutPrice}{P} % Put option price
\newcommand{\Greeks}[1]{\mathit{#1}} % Greek letters

% Risk measures
\newcommand{\VaR}{\mathrm{VaR}} % Value at Risk
\newcommand{\CVaR}{\mathrm{CVaR}} % Conditional VaR
\newcommand{\Sharpe}{\mathrm{SR}} % Sharpe Ratio

% Time series
\newcommand{\AR}{\mathrm{AR}} % Autoregressive
\newcommand{\MA}{\mathrm{MA}} % Moving average
\newcommand{\GARCH}{\mathrm{GARCH}} % GARCH

% Blockchain/Crypto
\newcommand{\Hash}{\mathrm{Hash}} % Hash function
\newcommand{\Block}{\mathcal{B}} % Block
\newcommand{\Chain}{\mathcal{C}} % Chain

% Real numbers, integers
\newcommand{\R}{\mathbb{R}}
\newcommand{\Z}{\mathbb{Z}}
\newcommand{\N}{\mathbb{N}}

% ======================= TIKZ STYLES =======================
% Styles for finance-related diagrams
\tikzstyle{process} = [rectangle, minimum width=3cm, minimum height=1cm, text centered, draw=dfblue, fill=dflightblue4, thick]
\tikzstyle{decision} = [diamond, minimum width=3cm, minimum height=1cm, text centered, draw=dfteal, fill=dflightblue4, thick]
\tikzstyle{arrow} = [thick,->,>=stealth,color=dfblue]
\tikzstyle{blockchain} = [rectangle, rounded corners, minimum width=2.5cm, minimum height=1cm, text centered, draw=dfteal, fill=dflightblue3, thick]
\tikzstyle{transaction} = [circle, minimum size=0.8cm, text centered, draw=dforange, fill=dflightblue4, thick]

% ======================= FOOTER TEMPLATE =======================
\setbeamertemplate{footline}{
    \hbox{\begin{beamercolorbox}[wd=\paperwidth,ht=2.5ex,dp=1ex,leftskip=.5em,rightskip=.5em]{author in head/foot}
    \tiny
    \textbf{Digital Finance} \hfill
    Joerg Osterrieder \hfill
    \insertdate \hfill
    Page \insertframenumber{} / \inserttotalframenumber
    \end{beamercolorbox}}
}

% ======================= SECTION DIVIDER TEMPLATE =======================
\AtBeginSection[]{
\begin{frame}[plain]
\vfill
\centering
\begin{beamercolorbox}[sep=12pt,center]{title}
\usebeamerfont{title}\LARGE\insertsection\par
\end{beamercolorbox}
\vfill
\end{frame}
}


% ======================= DOCUMENT INFO =======================
\title[Lecture Preview]{Platform Economics --- Lecture Overview}
\subtitle{What This Lecture Covers}
\author{Joerg Osterrieder}
\institute{Digital Finance}
\date{}

\begin{document}

% =======================================================================
% SLIDE 1: Opening question — Pipeline vs Platform
% =======================================================================
\begin{frame}{Why Do Some Businesses Connect Instead of Produce?}
\begin{columns}[T]
\begin{column}{0.55\textwidth}
\footnotesize
A bakery bakes bread and sells it. A ride-sharing app bakes nothing --- it connects a rider with a driver. Both are businesses, but they follow entirely different economic logic.

\vspace{2mm}
\begin{center}
\begin{tabular}{l l l}
\toprule
& \textbf{Pipeline} & \textbf{Platform} \\
\midrule
Does what? & Produces \& sells & Connects groups \\
Grows how? & Linear & Non-linear \\
Owns what? & Inventory, IP & User base, data \\
Protected by? & Cost, brand & Network effects \\
\bottomrule
\end{tabular}
\end{center}

\vspace{2mm}
\scriptsize
This distinction matters because it changes how businesses scale, price, compete, and fail. The lecture works through the economics behind each row of this table.
\end{column}
\begin{column}{0.42\textwidth}
% COMIC 1: Pipeline vs Platform — with personality
\begin{center}
\begin{tikzpicture}[scale=0.75]
% --- Pipeline scene ---
\node[font=\tiny\bfseries, mlpurple] at (1.5,4.3) {Pipeline};
% Factory with chimney
\draw[thick, mlgray] (0,2.8) rectangle (1.2,3.6);
\node[font=\tiny] at (0.6,3.2) {Factory};
\draw[thick, mlgray] (0.4,3.6) -- (0.4,3.9) -- (0.8,3.9) -- (0.8,3.6);
% Smoke puffs
\draw[mlgray, thick] (0.6,3.9) .. controls (0.5,4.1) and (0.7,4.15) .. (0.6,4.3);
% Arrow
\draw[->, thick, mlpurple] (1.4,3.2) -- (2.0,3.2);
% Box product
\draw[thick, mlgray] (2.1,2.9) rectangle (2.9,3.5);
\node[font=\tiny] at (2.5,3.2) {Product};
% Arrow
\draw[->, thick, mlpurple] (3.1,3.2) -- (3.5,3.2);
% Stick figure buyer
\draw[thick] (3.8,3.5) circle (0.15);
\draw[thick] (3.8,3.35) -- (3.8,2.9);
\draw[thick] (3.8,3.2) -- (3.55,3.0);
\draw[thick] (3.8,3.2) -- (4.05,3.0);
\draw[thick] (3.8,2.9) -- (3.6,2.6);
\draw[thick] (3.8,2.9) -- (4.0,2.6);

% --- Platform scene ---
\node[font=\tiny\bfseries, mlpurple] at (1.5,2.0) {Platform};
% Stick figure A (rider)
\draw[thick, dfteal] (0.3,1.5) circle (0.15);
\draw[thick, dfteal] (0.3,1.35) -- (0.3,0.9);
\draw[thick, dfteal] (0.3,1.2) -- (0.05,1.0);
\draw[thick, dfteal] (0.3,1.2) -- (0.55,1.0);
\draw[thick, dfteal] (0.3,0.9) -- (0.1,0.6);
\draw[thick, dfteal] (0.3,0.9) -- (0.5,0.6);
% Speech bubble
\draw[thick, mlgray, rounded corners=2pt] (0.55,1.45) rectangle (1.55,1.8);
\node[font=\tiny] at (1.05,1.63) {I need a ride};
\draw[thick, mlgray] (0.5,1.55) -- (0.55,1.6);

% Platform box — glowing
\draw[thick, mlpurple, fill=mllavender4, rounded corners=3pt] (1.7,0.8) rectangle (2.8,1.6);
\node[font=\tiny\bfseries, mlpurple] at (2.25,1.25) {App};
\node[font=\tiny, mlgray] at (2.25,1.05) {\textit{match!}};
% Arrows with labels
\draw[->, thick, dfteal] (0.7,1.1) -- (1.7,1.1);
\draw[->, thick, dfteal] (2.8,1.1) -- (3.4,1.1);

% Stick figure B (driver)
\draw[thick, dfteal] (3.7,1.5) circle (0.15);
\draw[thick, dfteal] (3.7,1.35) -- (3.7,0.9);
\draw[thick, dfteal] (3.7,1.2) -- (3.45,1.0);
\draw[thick, dfteal] (3.7,1.2) -- (3.95,1.0);
\draw[thick, dfteal] (3.7,0.9) -- (3.5,0.6);
\draw[thick, dfteal] (3.7,0.9) -- (3.9,0.6);
% Speech bubble
\draw[thick, mlgray, rounded corners=2pt] (2.55,1.45) rectangle (3.55,1.8);
\node[font=\tiny] at (3.05,1.63) {I have a car};
\draw[thick, mlgray] (3.6,1.55) -- (3.55,1.6);
\end{tikzpicture}
\end{center}
\end{column}
\end{columns}
\end{frame}

% =======================================================================
% SLIDE 2: Network Effects — why your group chat proves a point
% =======================================================================
\begin{frame}{Your Group Chat Proves a Point About Network Effects}
\begin{columns}[T]
\begin{column}{0.55\textwidth}
% CHART 1: Linear vs non-linear value growth
\begin{center}
\begin{tikzpicture}
\begin{axis}[
    width=7cm, height=4.5cm,
    xlabel={\scriptsize Number of users}, ylabel={\scriptsize Value},
    xmin=0, xmax=10, ymin=0, ymax=50,
    xtick={2,4,6,8,10}, ytick={0,10,20,30,40,50},
    legend style={at={(0.03,0.97)}, anchor=north west, font=\tiny},
    grid=major, grid style={gray!20},
    tick label style={font=\tiny}
]
\addplot[mlgray, thick, dashed, domain=0:10] {x*5};
\addlegendentry{Pipeline (linear)}
\addplot[mlpurple, ultra thick, domain=0:10] {x*(x-1)/2};
\addlegendentry{Platform (network)}
\end{axis}
\end{tikzpicture}
\end{center}

\vspace{1mm}
\scriptsize
A group chat with 3 friends has 3 possible one-on-one conversations. Add a fourth friend and it jumps to 6. Add a fifth: 10. Each new person adds more connections than the last. This is the same force that drives platform growth.
\end{column}
\begin{column}{0.42\textwidth}
% COMIC 2: Group chat growing
\begin{center}
\begin{tikzpicture}[scale=0.75]
% Scene: three people connected
% Person 1
\draw[thick, dfteal] (0.5,3.6) circle (0.15);
\draw[thick, dfteal] (0.5,3.45) -- (0.5,3.0);
\draw[thick, dfteal] (0.5,3.3) -- (0.25,3.1);
\draw[thick, dfteal] (0.5,3.3) -- (0.75,3.1);
\draw[thick, dfteal] (0.5,3.0) -- (0.3,2.7);
\draw[thick, dfteal] (0.5,3.0) -- (0.7,2.7);
% Person 2
\draw[thick, dfteal] (2.0,3.6) circle (0.15);
\draw[thick, dfteal] (2.0,3.45) -- (2.0,3.0);
\draw[thick, dfteal] (2.0,3.3) -- (1.75,3.1);
\draw[thick, dfteal] (2.0,3.3) -- (2.25,3.1);
\draw[thick, dfteal] (2.0,3.0) -- (1.8,2.7);
\draw[thick, dfteal] (2.0,3.0) -- (2.2,2.7);
% Person 3
\draw[thick, dfteal] (3.5,3.6) circle (0.15);
\draw[thick, dfteal] (3.5,3.45) -- (3.5,3.0);
\draw[thick, dfteal] (3.5,3.3) -- (3.25,3.1);
\draw[thick, dfteal] (3.5,3.3) -- (3.75,3.1);
\draw[thick, dfteal] (3.5,3.0) -- (3.3,2.7);
\draw[thick, dfteal] (3.5,3.0) -- (3.7,2.7);
% Connection lines — triangle
\draw[thick, mlpurple] (0.5,3.2) -- (2.0,3.2);
\draw[thick, mlpurple] (2.0,3.2) -- (3.5,3.2);
\draw[thick, mlpurple] (0.5,3.05) -- (3.5,3.05);
\node[font=\tiny\bfseries, mlpurple] at (2.0,2.45) {3 connections};

% Divider
\draw[dashed, mlgray] (0,2.2) -- (4.2,2.2);

% New person joining — 4 people, 6 connections
\draw[thick, dfteal] (0.3,1.8) circle (0.12);
\draw[thick, dfteal] (0.3,1.68) -- (0.3,1.35);
\draw[thick, dfteal] (1.4,1.8) circle (0.12);
\draw[thick, dfteal] (1.4,1.68) -- (1.4,1.35);
\draw[thick, dfteal] (2.6,1.8) circle (0.12);
\draw[thick, dfteal] (2.6,1.68) -- (2.6,1.35);
\draw[thick, mlorange] (3.7,1.8) circle (0.12);
\draw[thick, mlorange] (3.7,1.68) -- (3.7,1.35);
\node[font=\tiny\bfseries, mlorange] at (3.7,1.15) {new!};
% All 6 connections
\draw[thick, mlpurple!50] (0.3,1.65) -- (1.4,1.65);
\draw[thick, mlpurple!50] (0.3,1.6) -- (2.6,1.6);
\draw[thick, mlpurple!50] (0.3,1.55) -- (3.7,1.55);
\draw[thick, mlpurple!50] (1.4,1.6) -- (2.6,1.6);
\draw[thick, mlpurple!50] (1.4,1.55) -- (3.7,1.55);
\draw[thick, mlpurple!50] (2.6,1.55) -- (3.7,1.55);
\node[font=\tiny\bfseries, mlpurple] at (2.0,0.95) {6 connections};

% Arrow between scenes
\draw[->, ultra thick, mlorange] (4.0,2.95) -- (4.0,2.35);
\node[font=\tiny, mlorange, rotate=-90] at (4.3,2.65) {+1 user};
\end{tikzpicture}
\end{center}
\end{column}
\end{columns}

\vspace{1mm}
\begin{block}{}
\scriptsize \textbf{Four types in the lecture:} direct (same-side, like your group chat), indirect (cross-side, like cardholders and merchants), data (usage improves algorithms), and negative (congestion, spam, fraud).
\end{block}
\end{frame}

% =======================================================================
% SLIDE 3: The Chicken-and-Egg Problem
% =======================================================================
\begin{frame}{Day One: Nobody Shows Up --- Now What?}
\begin{columns}[T]
\begin{column}{0.55\textwidth}
\footnotesize
You launch a marketplace. On day one you have zero sellers and zero buyers. Sellers will not list if nobody is browsing. Buyers will not browse if nothing is listed. This \textbf{chicken-and-egg problem} faces every two-sided platform at launch.

\vspace{2mm}
\scriptsize
\textbf{Six strategies the lecture works through:}
\begin{enumerate}\compactlist
\item \textbf{Subsidize one side} --- pay people to show up
\item \textbf{Single-player mode} --- be useful alone first
\item \textbf{Seed supply} --- create the first listings yourself
\item \textbf{Piggyback} --- launch inside an existing community
\item \textbf{Marquee user} --- attract one big name that draws the rest
\item \textbf{Micro-market} --- dominate one small niche, then expand
\end{enumerate}

\vspace{1mm}
Each has trade-offs. The lecture covers when each works, when it backfires, and real cases of both.
\end{column}
\begin{column}{0.42\textwidth}
% COMIC 3: Chicken-and-egg — more expressive
\begin{center}
\begin{tikzpicture}[scale=0.75]
% Empty platform — with cobwebs
\draw[thick, mlpurple, fill=mllavender4, rounded corners=3pt] (1.2,2.0) rectangle (2.8,2.8);
\node[font=\tiny\bfseries, mlpurple] at (2.0,2.5) {Marketplace};
\node[font=\tiny, mlgray] at (2.0,2.2) {\textit{0 listings}};
% Cobweb detail
\draw[thin, mlgray!50] (1.2,2.8) -- (1.5,2.55);
\draw[thin, mlgray!50] (1.2,2.65) -- (1.4,2.55);

% Seller on left — turning away
\draw[thick, dfteal] (0.3,3.4) circle (0.15);
\draw[thick, dfteal] (0.3,3.25) -- (0.3,2.8);
\draw[thick, dfteal] (0.3,3.1) -- (0.05,2.9);
\draw[thick, dfteal] (0.3,3.1) -- (0.55,2.9);
\draw[thick, dfteal] (0.3,2.8) -- (0.1,2.5);
\draw[thick, dfteal] (0.3,2.8) -- (0.5,2.5);
% Thought bubble
\draw[thick, mlgray, rounded corners=2pt] (-0.4,3.65) rectangle (1.4,4.15);
\node[font=\tiny] at (0.5,3.97) {Zero buyers?};
\node[font=\tiny] at (0.5,3.77) {I'm not listing.};
\fill[mlgray] (0.35,3.58) circle (0.04);
\fill[mlgray] (0.38,3.62) circle (0.03);
% Leaving arrow
\draw[->, ultra thick, dfred] (0.15,2.4) -- (-0.2,1.8);
\node[font=\tiny\bfseries, dfred] at (-0.1,1.65) {gone};

% Buyer on right — turning away
\draw[thick, dfteal] (3.7,3.4) circle (0.15);
\draw[thick, dfteal] (3.7,3.25) -- (3.7,2.8);
\draw[thick, dfteal] (3.7,3.1) -- (3.45,2.9);
\draw[thick, dfteal] (3.7,3.1) -- (3.95,2.9);
\draw[thick, dfteal] (3.7,2.8) -- (3.5,2.5);
\draw[thick, dfteal] (3.7,2.8) -- (3.9,2.5);
% Thought bubble
\draw[thick, mlgray, rounded corners=2pt] (2.7,3.65) rectangle (4.5,4.15);
\node[font=\tiny] at (3.6,3.97) {Nothing here?};
\node[font=\tiny] at (3.6,3.77) {Waste of time.};
\fill[mlgray] (3.6,3.58) circle (0.04);
\fill[mlgray] (3.57,3.62) circle (0.03);
% Leaving arrow
\draw[->, ultra thick, dfred] (3.85,2.4) -- (4.2,1.8);
\node[font=\tiny\bfseries, dfred] at (4.15,1.65) {gone};

% Labels
\node[font=\tiny\bfseries, dfteal] at (0.3,2.35) {Seller};
\node[font=\tiny\bfseries, dfteal] at (3.7,2.35) {Buyer};

% Bottom punchline
\node[font=\tiny\bfseries, dfred] at (2.0,1.3) {Both sides wait. Nobody moves first.};
\end{tikzpicture}
\end{center}
\end{column}
\end{columns}
\end{frame}

% =======================================================================
% SLIDE 4: Critical Mass and Market Tipping
% =======================================================================
\begin{frame}{The Moment a Platform Becomes Unstoppable}
\begin{columns}[T]
\begin{column}{0.55\textwidth}
% CHART 2: S-curve with critical mass
\begin{center}
\begin{tikzpicture}
\begin{axis}[
    width=7cm, height=4.5cm,
    xlabel={\scriptsize Time}, ylabel={\scriptsize Users},
    xmin=0, xmax=10, ymin=0, ymax=100,
    xtick=\empty, ytick=\empty,
    grid=none,
    tick label style={font=\tiny}
]
\addplot[mlpurple, ultra thick, smooth, domain=0:10, samples=50] {100/(1+exp(-1.5*(x-4)))};
% Critical mass line
\draw[dashed, dfred, thick] (axis cs:4,0) -- (axis cs:4,100);
\node[font=\tiny\bfseries, dfred] at (axis cs:4.5,92) {Critical mass};
% Zone labels with more detail
\node[font=\tiny, mlgray] at (axis cs:1.8,20) {Burning cash.};
\node[font=\tiny, mlgray] at (axis cs:1.8,12) {Users leaving.};
\node[font=\tiny, mlgray] at (axis cs:1.8,4) {``Will this work?''};
\node[font=\tiny, mlpurple] at (axis cs:7.2,82) {Self-sustaining.};
\node[font=\tiny, mlpurple] at (axis cs:7.2,74) {Growth feeds growth.};
\node[font=\tiny, mlpurple] at (axis cs:7.2,66) {``Hard to stop.''};
\end{axis}
\end{tikzpicture}
\end{center}
\end{column}
\begin{column}{0.42\textwidth}
\footnotesize
\textbf{Critical mass}: the point where organic growth exceeds churn without subsidies.

\vspace{2mm}
\scriptsize
Before this point, the platform is fragile --- one bad quarter can trigger a death spiral. After it, network effects compound and growth becomes self-reinforcing.

\vspace{3mm}
\textbf{When does a market ``tip'' to one winner?}

\vspace{1mm}
All three conditions must hold:
\begin{enumerate}\compactlist
\item \textbf{Strong network effects} --- each new user meaningfully increases value
\item \textbf{High switching costs} --- leaving is painful
\item \textbf{Hard to multi-home} --- using two platforms at once is difficult
\end{enumerate}

\vspace{2mm}
If any condition is weak, competition survives. The lecture covers how to diagnose each.
\end{column}
\end{columns}
\end{frame}

% =======================================================================
% SLIDE 5: Business Models — follow the money
% =======================================================================
\begin{frame}{Follow the Money: How Platforms Actually Get Paid}
\begin{columns}[T]
\begin{column}{0.55\textwidth}
% CHART 3: Stacked bar — revenue model types
\begin{center}
\begin{adjustbox}{max width=\linewidth}
\begin{tikzpicture}
\begin{axis}[
    width=7cm, height=4.5cm,
    ybar stacked,
    bar width=14pt,
    symbolic x coords={Payment,Lending,Trading,Insurance,BaaS},
    xtick=data,
    x tick label style={font=\tiny, rotate=25, anchor=east},
    ylabel={\scriptsize Revenue mix (\%)},
    y tick label style={font=\tiny},
    ymin=0, ymax=105,
    ytick={0,25,50,75,100},
    legend style={at={(1.02,1.0)}, anchor=north west, font=\tiny},
    legend cell align={left},
    grid=none,
    enlarge x limits=0.15,
    area style
]
% Commission
\addplot[fill=mlpurple!80, draw=mlpurple] coordinates {(Payment,70) (Lending,60) (Trading,15) (Insurance,50) (BaaS,55)};
% Subscription
\addplot[fill=dfteal!60, draw=dfteal] coordinates {(Payment,10) (Lending,10) (Trading,25) (Insurance,15) (BaaS,35)};
% Data/Float
\addplot[fill=mlorange!60, draw=mlorange] coordinates {(Payment,15) (Lending,20) (Trading,50) (Insurance,25) (BaaS,5)};
% Other
\addplot[fill=mlgray!40, draw=mlgray] coordinates {(Payment,5) (Lending,10) (Trading,10) (Insurance,10) (BaaS,5)};
\legend{Commission, Subscription, Data/Float, Other}
\end{axis}
\end{tikzpicture}
\end{adjustbox}
\end{center}

\vspace{-1mm}
\scriptsize
\textit{Illustrative revenue mix by FinTech platform type.}\\
Notice: trading platforms earn most of their revenue from data and float --- not commissions. If you pay zero commission, someone else is paying.
\end{column}
\begin{column}{0.42\textwidth}
\footnotesize
\textbf{Four revenue models:}
\begin{description}\compactlist
\item[\scriptsize Commission] \% of each transaction
\item[\scriptsize Subscription] Monthly or annual access fee
\item[\scriptsize Freemium] Free basic tier, paid premium
\item[\scriptsize Data/Float] Selling order flow, earning interest on held funds
\end{description}

\vspace{3mm}
\scriptsize
\textbf{The health check --- three numbers:}
\begin{itemize}\compactlist
\item Does each customer pay back more than it cost to get them? (\textbf{LTV} $>$ 3$\times$ \textbf{CAC})
\item How long until they do? (payback $<$ 18 months)
\item How many leave each month? (churn $<$ 5\%)
\end{itemize}

\vspace{2mm}
The lecture uses these metrics to separate real businesses from venture-subsidized ones.
\end{column}
\end{columns}
\end{frame}

% =======================================================================
% SLIDE 6: Lecture roadmap
% =======================================================================
\begin{frame}{What the Lecture Covers: Eight Sections}
\footnotesize
\begin{center}
\begin{tabular}{c l l}
\toprule
\textbf{\#} & \textbf{Section} & \textbf{Central Question} \\
\midrule
1 & What is a Platform? & What makes a platform different from a regular business? \\
2 & Two-Sided Markets & Why do networks grow explosively --- or collapse? \\
3 & Strategy \& Competition & How do you launch when nobody is there yet? \\
4 & Business Models & Where does the money actually come from? \\
5 & Data, Trust, Governance & Who controls the platform, and why does it matter? \\
6 & Regulation \& Failures & What kills a platform, and can regulation prevent it? \\
7 & Platforms in Finance & How does this apply to payments, lending, and trading? \\
8 & Synthesis & What happens when platforms decentralize? \\
\bottomrule
\end{tabular}
\end{center}

\vspace{3mm}
\begin{columns}[T]
\begin{column}{0.48\textwidth}
\begin{block}{Lecture format}
\begin{itemize}\compactlist
\item 59 content frames, 13 diagrams
\item 4 in-class discussion exercises
\item 2 self-assessment quizzes
\item 1 group workshop (you present)
\end{itemize}
\end{block}
\end{column}
\begin{column}{0.48\textwidth}
\begin{block}{Academic references}
\scriptsize
\begin{itemize}\compactlist
\item Rochet-Tirole (2003/2006)
\item Parker, Van Alstyne, Choudary (2016)
\item Eisenmann, Parker, Van Alstyne (2006)
\item Evans and Schmalensee (2016)
\end{itemize}
\end{block}
\end{column}
\end{columns}
\end{frame}

\end{document}
