% Platform Economics: Pre-Class Intro (3 slides)
% Shows main output and learning achievements after the lecture
\documentclass[11pt,aspectratio=169]{beamer}
\usetheme{Madrid}

% ======================= PACKAGES =======================
\usepackage{graphicx}
\usepackage{booktabs}
\usepackage{adjustbox}
\usepackage{multicol}
\usepackage{amsmath}
\usepackage{amssymb}
\usepackage{tikz}
\usetikzlibrary{arrows,shapes,positioning,shadows,trees}
\usepackage{listings}
\usepackage{xcolor}

% ======================= COLOR DEFINITIONS =======================
% Primary color scheme: Blue/Teal for Digital Finance
\definecolor{dfblue}{RGB}{0,102,204}
\definecolor{dfteal}{RGB}{0,153,153}
\definecolor{dfcyan}{RGB}{51,187,204}
\definecolor{dflightblue}{RGB}{153,204,255}
\definecolor{dflightblue2}{RGB}{173,214,255}
\definecolor{dflightblue3}{RGB}{193,224,255}
\definecolor{dflightblue4}{RGB}{213,234,255}

% Accent colors for finance applications
\definecolor{dfgreen}{RGB}{44, 160, 44}
\definecolor{dfred}{RGB}{214, 39, 40}
\definecolor{dforange}{RGB}{255, 127, 14}
\definecolor{dfgray}{RGB}{127, 127, 127}

% Utility colors
\definecolor{lightgray}{RGB}{240, 240, 240}
\definecolor{midgray}{RGB}{180, 180, 180}
\definecolor{codebg}{RGB}{245, 245, 245}

% ======================= THEME CUSTOMIZATION =======================
% Apply Digital Finance color scheme to Madrid theme
\setbeamercolor{palette primary}{bg=dflightblue3,fg=dfblue}
\setbeamercolor{palette secondary}{bg=dflightblue2,fg=dfblue}
\setbeamercolor{palette tertiary}{bg=dfteal,fg=white}
\setbeamercolor{palette quaternary}{bg=dfblue,fg=white}

\setbeamercolor{structure}{fg=dfblue}
\setbeamercolor{section in toc}{fg=dfblue}
\setbeamercolor{subsection in toc}{fg=dfteal}
\setbeamercolor{title}{fg=dfblue}
\setbeamercolor{frametitle}{fg=dfblue,bg=dflightblue3}
\setbeamercolor{block title}{bg=dflightblue2,fg=dfblue}
\setbeamercolor{block body}{bg=dflightblue4,fg=black}

% Remove navigation symbols for cleaner look
\setbeamertemplate{navigation symbols}{}

% Clean itemize/enumerate
\setbeamertemplate{itemize items}[circle]
\setbeamertemplate{enumerate items}[default]

% Margins for readability
\setbeamersize{text margin left=8mm,text margin right=8mm}

% ======================= LISTINGS CONFIGURATION =======================
% Python code style
\lstdefinestyle{pythonstyle}{
    language=Python,
    basicstyle=\ttfamily\footnotesize,
    keywordstyle=\color{dfblue}\bfseries,
    stringstyle=\color{dforange},
    commentstyle=\color{dfgray}\itshape,
    numberstyle=\tiny\color{dfgray},
    numbers=left,
    numbersep=5pt,
    backgroundcolor=\color{codebg},
    showspaces=false,
    showstringspaces=false,
    showtabs=false,
    frame=single,
    rulecolor=\color{midgray},
    tabsize=4,
    captionpos=b,
    breaklines=true,
    breakatwhitespace=false,
    escapeinside={(*@}{@*)},
    xleftmargin=10pt,
    xrightmargin=10pt
}

% Solidity code style
\lstdefinestyle{soliditystyle}{
    language=Java, % closest approximation
    basicstyle=\ttfamily\footnotesize,
    keywordstyle=\color{dfteal}\bfseries,
    stringstyle=\color{dforange},
    commentstyle=\color{dfgray}\itshape,
    numberstyle=\tiny\color{dfgray},
    numbers=left,
    numbersep=5pt,
    backgroundcolor=\color{codebg},
    showspaces=false,
    showstringspaces=false,
    showtabs=false,
    frame=single,
    rulecolor=\color{midgray},
    tabsize=2,
    captionpos=b,
    breaklines=true,
    breakatwhitespace=false,
    escapeinside={(*@}{@*)},
    xleftmargin=10pt,
    xrightmargin=10pt,
    morekeywords={pragma, contract, function, returns, public, private, view, pure, payable, address, uint256, mapping, event, modifier}
}

% Inline code command
\newcommand{\code}[1]{\texttt{\color{dfblue}#1}}

% ======================= CUSTOM COMMANDS =======================
% Bottom annotation (Madrid-style)
\newcommand{\bottomnote}[1]{%
\vfill
\vspace{-2mm}
\textcolor{dflightblue2}{\rule{\textwidth}{0.4pt}}
\vspace{1mm}
\footnotesize
\textbf{#1}
}

% Compact list spacing
\newcommand{\compactlist}{%
\setlength{\itemsep}{0pt}%
\setlength{\parskip}{0pt}%
\setlength{\parsep}{0pt}%
}

% Chart placeholder
\newcommand{\chartplaceholder}[2][5cm]{%
\begin{center}
\begin{adjustbox}{max width=0.95\textwidth, max height=#1}
\framebox[\textwidth][c]{%
\rule{0pt}{#1}%
\textcolor{midgray}{[#2]}%
}
\end{adjustbox}
\end{center}
}

% ======================= FINANCE NOTATION MACROS =======================
% Probability and statistics
\newcommand{\E}{\mathbb{E}} % Expected value
\newcommand{\Var}{\mathrm{Var}} % Variance
\newcommand{\Cov}{\mathrm{Cov}} % Covariance
\newcommand{\Prob}{\mathbb{P}} % Probability

% Distributions
\newcommand{\Normal}{\mathcal{N}} % Normal distribution
\newcommand{\Uniform}{\mathcal{U}} % Uniform distribution

% Returns and prices
\newcommand{\Ret}{R} % Return
\newcommand{\LogRet}{r} % Log return
\newcommand{\Price}{S} % Price/Stock price
\newcommand{\Strike}{K} % Strike price

% Options and derivatives
\newcommand{\CallPrice}{C} % Call option price
\newcommand{\PutPrice}{P} % Put option price
\newcommand{\Greeks}[1]{\mathit{#1}} % Greek letters

% Risk measures
\newcommand{\VaR}{\mathrm{VaR}} % Value at Risk
\newcommand{\CVaR}{\mathrm{CVaR}} % Conditional VaR
\newcommand{\Sharpe}{\mathrm{SR}} % Sharpe Ratio

% Time series
\newcommand{\AR}{\mathrm{AR}} % Autoregressive
\newcommand{\MA}{\mathrm{MA}} % Moving average
\newcommand{\GARCH}{\mathrm{GARCH}} % GARCH

% Blockchain/Crypto
\newcommand{\Hash}{\mathrm{Hash}} % Hash function
\newcommand{\Block}{\mathcal{B}} % Block
\newcommand{\Chain}{\mathcal{C}} % Chain

% Real numbers, integers
\newcommand{\R}{\mathbb{R}}
\newcommand{\Z}{\mathbb{Z}}
\newcommand{\N}{\mathbb{N}}

% ======================= TIKZ STYLES =======================
% Styles for finance-related diagrams
\tikzstyle{process} = [rectangle, minimum width=3cm, minimum height=1cm, text centered, draw=dfblue, fill=dflightblue4, thick]
\tikzstyle{decision} = [diamond, minimum width=3cm, minimum height=1cm, text centered, draw=dfteal, fill=dflightblue4, thick]
\tikzstyle{arrow} = [thick,->,>=stealth,color=dfblue]
\tikzstyle{blockchain} = [rectangle, rounded corners, minimum width=2.5cm, minimum height=1cm, text centered, draw=dfteal, fill=dflightblue3, thick]
\tikzstyle{transaction} = [circle, minimum size=0.8cm, text centered, draw=dforange, fill=dflightblue4, thick]

% ======================= FOOTER TEMPLATE =======================
\setbeamertemplate{footline}{
    \hbox{\begin{beamercolorbox}[wd=\paperwidth,ht=2.5ex,dp=1ex,leftskip=.5em,rightskip=.5em]{author in head/foot}
    \tiny
    \textbf{Digital Finance} \hfill
    Joerg Osterrieder \hfill
    \insertdate \hfill
    Page \insertframenumber{} / \inserttotalframenumber
    \end{beamercolorbox}}
}

% ======================= SECTION DIVIDER TEMPLATE =======================
\AtBeginSection[]{
\begin{frame}[plain]
\vfill
\centering
\begin{beamercolorbox}[sep=12pt,center]{title}
\usebeamerfont{title}\LARGE\insertsection\par
\end{beamercolorbox}
\vfill
\end{frame}
}


% ======================= DOCUMENT INFO =======================
\title[Lecture Preview]{Platform Economics --- What You Will Learn}
\subtitle{A Preview of the Lecture Outcomes}
\author{Joerg Osterrieder}
\institute{Digital Finance}
\date{}

\begin{document}

% =======================================================================
% SLIDE 1: The Big Picture
% =======================================================================
\begin{frame}{Platform Economics: The Big Picture}

\begin{alertblock}{After this lecture, you will see digital markets differently.}
\end{alertblock}

\vspace{2mm}
\footnotesize
Platforms are the dominant business model of the digital economy. They power payments, lending, trading, and insurance --- yet most people cannot explain \emph{why} they work or \emph{when} they fail.

\vspace{3mm}
\textbf{\color{mlpurple}This lecture gives you a complete analytical toolkit:}

\vspace{2mm}
\begin{columns}[T]
\begin{column}{0.48\textwidth}
\begin{block}{Theory}
\begin{itemize}\compactlist
\item What makes a platform different from a traditional business
\item How network effects create explosive growth --- and collapse
\item Why some markets tip to one winner while others stay competitive
\end{itemize}
\end{block}
\end{column}
\begin{column}{0.48\textwidth}
\begin{block}{Practice}
\begin{itemize}\compactlist
\item How platforms price, launch, and compete
\item How to evaluate whether a FinTech has a real moat
\item How regulation and decentralization are reshaping the landscape
\end{itemize}
\end{block}
\end{column}
\end{columns}

\vspace{3mm}
\begin{exampleblock}{Format}
\scriptsize 8 sections | 59 frames | 13 diagrams | 4 case discussions | 2 self-assessments | 1 group workshop
\end{exampleblock}
\end{frame}

% =======================================================================
% SLIDE 2: Six Things You Will Be Able To Do
% =======================================================================
\begin{frame}{After the Lecture: Six Things You Will Be Able To Do}
\footnotesize

\begin{enumerate}\compactlist
\item \textbf{Distinguish platforms from pipelines} --- instantly classify any business by asking: ``Does it \emph{connect} or does it \emph{produce}?''

\vspace{1mm}
\item \textbf{Identify all four types of network effects} --- direct, indirect, data, and negative --- and explain how each shapes competition

\vspace{1mm}
\item \textbf{Diagnose the chicken-and-egg problem} and pick from six proven launch strategies (subsidize, single-player mode, seed supply, piggyback, marquee user, micro-market)

\vspace{1mm}
\item \textbf{Evaluate unit economics} using CAC, LTV, payback period, and churn --- and spot when growth is real vs.\ venture-subsidized

\vspace{1mm}
\item \textbf{Apply a six-question framework} to any FinTech platform: network effects? unit economics? switching costs? data advantage? defensibility? regulation?

\vspace{1mm}
\item \textbf{Debate the future}: centralized vs.\ decentralized platforms, token economics, AI-driven network effects, and the regulator's dilemma
\end{enumerate}

\vspace{2mm}
\begin{alertblock}{The Core Skill}
\scriptsize Analyze any platform business and assess whether it has a sustainable competitive advantage --- or just expensive user acquisition.
\end{alertblock}
\end{frame}

% =======================================================================
% SLIDE 3: Key Frameworks You Will Take Away
% =======================================================================
\begin{frame}{Key Frameworks You Will Take Away}
\footnotesize

\begin{columns}[T]
\begin{column}{0.48\textwidth}

\textbf{\color{mlpurple}1. Platform Classification}\\[1mm]
\scriptsize Pipeline vs.\ platform; transaction, innovation, investment, integrated types.

\vspace{3mm}
\textbf{\color{mlpurple}2. Network Effects Toolkit}\\[1mm]
\scriptsize Four types, critical mass S-curve, winner-take-most conditions (network effects + switching costs + low multi-homing).

\vspace{3mm}
\textbf{\color{mlpurple}3. Pricing Logic}\\[1mm]
\scriptsize Subsidy side vs.\ money side; Rochet-Tirole pricing principle; access, usage, and enhanced fees.

\end{column}
\begin{column}{0.48\textwidth}

\textbf{\color{mlpurple}4. Business Model Assessment}\\[1mm]
\scriptsize Commission, subscription, freemium, advertising, PFOF, float; unit economics health check.

\vspace{3mm}
\textbf{\color{mlpurple}5. Data Flywheel}\\[1mm]
\scriptsize More users $\rightarrow$ more data $\rightarrow$ better algorithms $\rightarrow$ better experience $\rightarrow$ more users. Why this moat compounds.

\vspace{3mm}
\textbf{\color{mlpurple}6. Sustainability Test}\\[1mm]
\scriptsize Six-question evaluation framework applicable to any FinTech you encounter --- in this course or in your career.

\end{column}
\end{columns}

\vspace{4mm}
\begin{exampleblock}{Ready?}
\scriptsize These frameworks will let you move from ``I use platforms'' to ``I understand how platforms work, compete, and fail.''
\end{exampleblock}

\bottomnote{Lecture based on Rochet-Tirole (2003/2006), Parker-Van Alstyne-Choudary (2016), Eisenmann-Parker-Van Alstyne (2006), Evans-Schmalensee (2016).}
\end{frame}

\end{document}
