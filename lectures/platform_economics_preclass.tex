% Platform Economics: Pre-Class Discovery Handout
% 2-page handout to prime student thinking before the lecture
\documentclass[11pt,aspectratio=169]{beamer}
\usetheme{Madrid}

% ======================= PACKAGES =======================
\usepackage{graphicx}
\usepackage{booktabs}
\usepackage{adjustbox}
\usepackage{multicol}
\usepackage{amsmath}
\usepackage{amssymb}
\usepackage{tikz}
\usetikzlibrary{arrows,shapes,positioning,shadows,trees}
\usepackage{listings}
\usepackage{xcolor}

% ======================= COLOR DEFINITIONS =======================
% Primary color scheme: Blue/Teal for Digital Finance
\definecolor{dfblue}{RGB}{0,102,204}
\definecolor{dfteal}{RGB}{0,153,153}
\definecolor{dfcyan}{RGB}{51,187,204}
\definecolor{dflightblue}{RGB}{153,204,255}
\definecolor{dflightblue2}{RGB}{173,214,255}
\definecolor{dflightblue3}{RGB}{193,224,255}
\definecolor{dflightblue4}{RGB}{213,234,255}

% Accent colors for finance applications
\definecolor{dfgreen}{RGB}{44, 160, 44}
\definecolor{dfred}{RGB}{214, 39, 40}
\definecolor{dforange}{RGB}{255, 127, 14}
\definecolor{dfgray}{RGB}{127, 127, 127}

% Utility colors
\definecolor{lightgray}{RGB}{240, 240, 240}
\definecolor{midgray}{RGB}{180, 180, 180}
\definecolor{codebg}{RGB}{245, 245, 245}

% ======================= THEME CUSTOMIZATION =======================
% Apply Digital Finance color scheme to Madrid theme
\setbeamercolor{palette primary}{bg=dflightblue3,fg=dfblue}
\setbeamercolor{palette secondary}{bg=dflightblue2,fg=dfblue}
\setbeamercolor{palette tertiary}{bg=dfteal,fg=white}
\setbeamercolor{palette quaternary}{bg=dfblue,fg=white}

\setbeamercolor{structure}{fg=dfblue}
\setbeamercolor{section in toc}{fg=dfblue}
\setbeamercolor{subsection in toc}{fg=dfteal}
\setbeamercolor{title}{fg=dfblue}
\setbeamercolor{frametitle}{fg=dfblue,bg=dflightblue3}
\setbeamercolor{block title}{bg=dflightblue2,fg=dfblue}
\setbeamercolor{block body}{bg=dflightblue4,fg=black}

% Remove navigation symbols for cleaner look
\setbeamertemplate{navigation symbols}{}

% Clean itemize/enumerate
\setbeamertemplate{itemize items}[circle]
\setbeamertemplate{enumerate items}[default]

% Margins for readability
\setbeamersize{text margin left=8mm,text margin right=8mm}

% ======================= LISTINGS CONFIGURATION =======================
% Python code style
\lstdefinestyle{pythonstyle}{
    language=Python,
    basicstyle=\ttfamily\footnotesize,
    keywordstyle=\color{dfblue}\bfseries,
    stringstyle=\color{dforange},
    commentstyle=\color{dfgray}\itshape,
    numberstyle=\tiny\color{dfgray},
    numbers=left,
    numbersep=5pt,
    backgroundcolor=\color{codebg},
    showspaces=false,
    showstringspaces=false,
    showtabs=false,
    frame=single,
    rulecolor=\color{midgray},
    tabsize=4,
    captionpos=b,
    breaklines=true,
    breakatwhitespace=false,
    escapeinside={(*@}{@*)},
    xleftmargin=10pt,
    xrightmargin=10pt
}

% Solidity code style
\lstdefinestyle{soliditystyle}{
    language=Java, % closest approximation
    basicstyle=\ttfamily\footnotesize,
    keywordstyle=\color{dfteal}\bfseries,
    stringstyle=\color{dforange},
    commentstyle=\color{dfgray}\itshape,
    numberstyle=\tiny\color{dfgray},
    numbers=left,
    numbersep=5pt,
    backgroundcolor=\color{codebg},
    showspaces=false,
    showstringspaces=false,
    showtabs=false,
    frame=single,
    rulecolor=\color{midgray},
    tabsize=2,
    captionpos=b,
    breaklines=true,
    breakatwhitespace=false,
    escapeinside={(*@}{@*)},
    xleftmargin=10pt,
    xrightmargin=10pt,
    morekeywords={pragma, contract, function, returns, public, private, view, pure, payable, address, uint256, mapping, event, modifier}
}

% Inline code command
\newcommand{\code}[1]{\texttt{\color{dfblue}#1}}

% ======================= CUSTOM COMMANDS =======================
% Bottom annotation (Madrid-style)
\newcommand{\bottomnote}[1]{%
\vfill
\vspace{-2mm}
\textcolor{dflightblue2}{\rule{\textwidth}{0.4pt}}
\vspace{1mm}
\footnotesize
\textbf{#1}
}

% Compact list spacing
\newcommand{\compactlist}{%
\setlength{\itemsep}{0pt}%
\setlength{\parskip}{0pt}%
\setlength{\parsep}{0pt}%
}

% Chart placeholder
\newcommand{\chartplaceholder}[2][5cm]{%
\begin{center}
\begin{adjustbox}{max width=0.95\textwidth, max height=#1}
\framebox[\textwidth][c]{%
\rule{0pt}{#1}%
\textcolor{midgray}{[#2]}%
}
\end{adjustbox}
\end{center}
}

% ======================= FINANCE NOTATION MACROS =======================
% Probability and statistics
\newcommand{\E}{\mathbb{E}} % Expected value
\newcommand{\Var}{\mathrm{Var}} % Variance
\newcommand{\Cov}{\mathrm{Cov}} % Covariance
\newcommand{\Prob}{\mathbb{P}} % Probability

% Distributions
\newcommand{\Normal}{\mathcal{N}} % Normal distribution
\newcommand{\Uniform}{\mathcal{U}} % Uniform distribution

% Returns and prices
\newcommand{\Ret}{R} % Return
\newcommand{\LogRet}{r} % Log return
\newcommand{\Price}{S} % Price/Stock price
\newcommand{\Strike}{K} % Strike price

% Options and derivatives
\newcommand{\CallPrice}{C} % Call option price
\newcommand{\PutPrice}{P} % Put option price
\newcommand{\Greeks}[1]{\mathit{#1}} % Greek letters

% Risk measures
\newcommand{\VaR}{\mathrm{VaR}} % Value at Risk
\newcommand{\CVaR}{\mathrm{CVaR}} % Conditional VaR
\newcommand{\Sharpe}{\mathrm{SR}} % Sharpe Ratio

% Time series
\newcommand{\AR}{\mathrm{AR}} % Autoregressive
\newcommand{\MA}{\mathrm{MA}} % Moving average
\newcommand{\GARCH}{\mathrm{GARCH}} % GARCH

% Blockchain/Crypto
\newcommand{\Hash}{\mathrm{Hash}} % Hash function
\newcommand{\Block}{\mathcal{B}} % Block
\newcommand{\Chain}{\mathcal{C}} % Chain

% Real numbers, integers
\newcommand{\R}{\mathbb{R}}
\newcommand{\Z}{\mathbb{Z}}
\newcommand{\N}{\mathbb{N}}

% ======================= TIKZ STYLES =======================
% Styles for finance-related diagrams
\tikzstyle{process} = [rectangle, minimum width=3cm, minimum height=1cm, text centered, draw=dfblue, fill=dflightblue4, thick]
\tikzstyle{decision} = [diamond, minimum width=3cm, minimum height=1cm, text centered, draw=dfteal, fill=dflightblue4, thick]
\tikzstyle{arrow} = [thick,->,>=stealth,color=dfblue]
\tikzstyle{blockchain} = [rectangle, rounded corners, minimum width=2.5cm, minimum height=1cm, text centered, draw=dfteal, fill=dflightblue3, thick]
\tikzstyle{transaction} = [circle, minimum size=0.8cm, text centered, draw=dforange, fill=dflightblue4, thick]

% ======================= FOOTER TEMPLATE =======================
\setbeamertemplate{footline}{
    \hbox{\begin{beamercolorbox}[wd=\paperwidth,ht=2.5ex,dp=1ex,leftskip=.5em,rightskip=.5em]{author in head/foot}
    \tiny
    \textbf{Digital Finance} \hfill
    Joerg Osterrieder \hfill
    \insertdate \hfill
    Page \insertframenumber{} / \inserttotalframenumber
    \end{beamercolorbox}}
}

% ======================= SECTION DIVIDER TEMPLATE =======================
\AtBeginSection[]{
\begin{frame}[plain]
\vfill
\centering
\begin{beamercolorbox}[sep=12pt,center]{title}
\usebeamerfont{title}\LARGE\insertsection\par
\end{beamercolorbox}
\vfill
\end{frame}
}


% ======================= DOCUMENT INFO =======================
\title[Pre-Class Discovery]{Platform Economics --- Pre-Class Discovery}
\subtitle{Think About These Before We Meet}
\author{Joerg Osterrieder}
\institute{Digital Finance}
\date{}

\begin{document}

% =======================================================================
% PAGE 1: Everyday Platforms -- What Do You Already Know?
% =======================================================================
\begin{frame}{Everyday Platforms: What Do You Already Know?}

\begin{alertblock}{You already use platforms every day --- but have you ever thought about \emph{why} they work?}
\end{alertblock}

\vspace{2mm}
\footnotesize
\textbf{\color{mlpurple}Discovery Activity 1: Your Platform Diary}

Think about the last 48 hours. List every digital service you used where \textbf{two different groups} interact through the same app or website (e.g., buyers and sellers, riders and drivers, borrowers and lenders).

\vspace{2mm}
\begin{columns}[T]
\begin{column}{0.48\textwidth}
\begin{block}{For each service, note:}
\begin{itemize}\compactlist
\item Who are the two (or more) sides?
\item Does the service \emph{make} something, or does it \emph{connect} people?
\item Could one side exist without the other?
\item Did you pay? If not, who does?
\end{itemize}
\end{block}
\end{column}
\begin{column}{0.48\textwidth}
\begin{block}{Then ask yourself:}
\begin{itemize}\compactlist
\item Why do you use \emph{this} service and not an alternative?
\item Would it be easy to switch? Why or why not?
\item Is the service more useful because many people use it?
\item What would happen if half the users left overnight?
\end{itemize}
\end{block}
\end{column}
\end{columns}

\vspace{3mm}
\begin{exampleblock}{Bring to class}
\scriptsize Your list of 3--5 services and your answers. We will use these as examples throughout the lecture.
\end{exampleblock}

\bottomnote{No right or wrong answers --- the goal is to notice patterns you already experience but may not have named yet.}
\end{frame}

% =======================================================================
% PAGE 2: Three Puzzles to Ponder
% =======================================================================
\begin{frame}{Three Puzzles to Ponder}
\footnotesize

\textbf{\color{mlpurple}Puzzle 1: The Empty Marketplace}
\begin{quote}
\scriptsize
You want to launch a new online marketplace connecting local farmers with restaurant chefs. On day one, you have zero farmers and zero chefs. Farmers won't list produce if no chefs are browsing. Chefs won't browse if no produce is listed. \textbf{How do you escape this trap?} Write down at least two ideas.
\end{quote}

\vspace{2mm}
\textbf{\color{mlpurple}Puzzle 2: The Free Product}
\begin{quote}
\scriptsize
A popular financial app lets you buy and sell stocks with zero commission. The company has thousands of employees, expensive offices, and sophisticated technology. \textbf{If you are not paying, how does the company make money?} Who \emph{is} paying, and why? Is ``free'' really free?
\end{quote}

\vspace{2mm}
\textbf{\color{mlpurple}Puzzle 3: The Unbeatable Incumbent}
\begin{quote}
\scriptsize
A dominant payment network is accepted at virtually every store. A new competitor offers lower fees and better technology. Yet almost no merchants switch. \textbf{Why not?} What invisible force keeps merchants locked in, even when a cheaper option exists?
\end{quote}

\vspace{2mm}
\begin{alertblock}{Come prepared}
\scriptsize Jot down your best guesses. In the lecture, we will name the economic forces behind each puzzle and give you frameworks to analyze any platform business.
\end{alertblock}
\end{frame}

\end{document}
