% Smart Contracts: 10-Slide Mini-Lecture
% Standalone introduction following the 10-slide arc
\documentclass[11pt,aspectratio=169]{beamer}
\usetheme{Madrid}

% ======================= PACKAGES =======================
\usepackage{graphicx}
\usepackage{booktabs}
\usepackage{adjustbox}
\usepackage{multicol}
\usepackage{amsmath}
\usepackage{amssymb}
\usepackage{tikz}
\usetikzlibrary{arrows,shapes,positioning,shadows,trees}
\usepackage{listings}
\usepackage{xcolor}

% ======================= COLOR DEFINITIONS =======================
% Primary color scheme: Blue/Teal for Digital Finance
\definecolor{dfblue}{RGB}{0,102,204}
\definecolor{dfteal}{RGB}{0,153,153}
\definecolor{dfcyan}{RGB}{51,187,204}
\definecolor{dflightblue}{RGB}{153,204,255}
\definecolor{dflightblue2}{RGB}{173,214,255}
\definecolor{dflightblue3}{RGB}{193,224,255}
\definecolor{dflightblue4}{RGB}{213,234,255}

% Accent colors for finance applications
\definecolor{dfgreen}{RGB}{44, 160, 44}
\definecolor{dfred}{RGB}{214, 39, 40}
\definecolor{dforange}{RGB}{255, 127, 14}
\definecolor{dfgray}{RGB}{127, 127, 127}

% Utility colors
\definecolor{lightgray}{RGB}{240, 240, 240}
\definecolor{midgray}{RGB}{180, 180, 180}
\definecolor{codebg}{RGB}{245, 245, 245}

% ======================= THEME CUSTOMIZATION =======================
% Apply Digital Finance color scheme to Madrid theme
\setbeamercolor{palette primary}{bg=dflightblue3,fg=dfblue}
\setbeamercolor{palette secondary}{bg=dflightblue2,fg=dfblue}
\setbeamercolor{palette tertiary}{bg=dfteal,fg=white}
\setbeamercolor{palette quaternary}{bg=dfblue,fg=white}

\setbeamercolor{structure}{fg=dfblue}
\setbeamercolor{section in toc}{fg=dfblue}
\setbeamercolor{subsection in toc}{fg=dfteal}
\setbeamercolor{title}{fg=dfblue}
\setbeamercolor{frametitle}{fg=dfblue,bg=dflightblue3}
\setbeamercolor{block title}{bg=dflightblue2,fg=dfblue}
\setbeamercolor{block body}{bg=dflightblue4,fg=black}

% Remove navigation symbols for cleaner look
\setbeamertemplate{navigation symbols}{}

% Clean itemize/enumerate
\setbeamertemplate{itemize items}[circle]
\setbeamertemplate{enumerate items}[default]

% Margins for readability
\setbeamersize{text margin left=8mm,text margin right=8mm}

% ======================= LISTINGS CONFIGURATION =======================
% Python code style
\lstdefinestyle{pythonstyle}{
    language=Python,
    basicstyle=\ttfamily\footnotesize,
    keywordstyle=\color{dfblue}\bfseries,
    stringstyle=\color{dforange},
    commentstyle=\color{dfgray}\itshape,
    numberstyle=\tiny\color{dfgray},
    numbers=left,
    numbersep=5pt,
    backgroundcolor=\color{codebg},
    showspaces=false,
    showstringspaces=false,
    showtabs=false,
    frame=single,
    rulecolor=\color{midgray},
    tabsize=4,
    captionpos=b,
    breaklines=true,
    breakatwhitespace=false,
    escapeinside={(*@}{@*)},
    xleftmargin=10pt,
    xrightmargin=10pt
}

% Solidity code style
\lstdefinestyle{soliditystyle}{
    language=Java, % closest approximation
    basicstyle=\ttfamily\footnotesize,
    keywordstyle=\color{dfteal}\bfseries,
    stringstyle=\color{dforange},
    commentstyle=\color{dfgray}\itshape,
    numberstyle=\tiny\color{dfgray},
    numbers=left,
    numbersep=5pt,
    backgroundcolor=\color{codebg},
    showspaces=false,
    showstringspaces=false,
    showtabs=false,
    frame=single,
    rulecolor=\color{midgray},
    tabsize=2,
    captionpos=b,
    breaklines=true,
    breakatwhitespace=false,
    escapeinside={(*@}{@*)},
    xleftmargin=10pt,
    xrightmargin=10pt,
    morekeywords={pragma, contract, function, returns, public, private, view, pure, payable, address, uint256, mapping, event, modifier}
}

% Inline code command
\newcommand{\code}[1]{\texttt{\color{dfblue}#1}}

% ======================= CUSTOM COMMANDS =======================
% Bottom annotation (Madrid-style)
\newcommand{\bottomnote}[1]{%
\vfill
\vspace{-2mm}
\textcolor{dflightblue2}{\rule{\textwidth}{0.4pt}}
\vspace{1mm}
\footnotesize
\textbf{#1}
}

% Compact list spacing
\newcommand{\compactlist}{%
\setlength{\itemsep}{0pt}%
\setlength{\parskip}{0pt}%
\setlength{\parsep}{0pt}%
}

% Chart placeholder
\newcommand{\chartplaceholder}[2][5cm]{%
\begin{center}
\begin{adjustbox}{max width=0.95\textwidth, max height=#1}
\framebox[\textwidth][c]{%
\rule{0pt}{#1}%
\textcolor{midgray}{[#2]}%
}
\end{adjustbox}
\end{center}
}

% ======================= FINANCE NOTATION MACROS =======================
% Probability and statistics
\newcommand{\E}{\mathbb{E}} % Expected value
\newcommand{\Var}{\mathrm{Var}} % Variance
\newcommand{\Cov}{\mathrm{Cov}} % Covariance
\newcommand{\Prob}{\mathbb{P}} % Probability

% Distributions
\newcommand{\Normal}{\mathcal{N}} % Normal distribution
\newcommand{\Uniform}{\mathcal{U}} % Uniform distribution

% Returns and prices
\newcommand{\Ret}{R} % Return
\newcommand{\LogRet}{r} % Log return
\newcommand{\Price}{S} % Price/Stock price
\newcommand{\Strike}{K} % Strike price

% Options and derivatives
\newcommand{\CallPrice}{C} % Call option price
\newcommand{\PutPrice}{P} % Put option price
\newcommand{\Greeks}[1]{\mathit{#1}} % Greek letters

% Risk measures
\newcommand{\VaR}{\mathrm{VaR}} % Value at Risk
\newcommand{\CVaR}{\mathrm{CVaR}} % Conditional VaR
\newcommand{\Sharpe}{\mathrm{SR}} % Sharpe Ratio

% Time series
\newcommand{\AR}{\mathrm{AR}} % Autoregressive
\newcommand{\MA}{\mathrm{MA}} % Moving average
\newcommand{\GARCH}{\mathrm{GARCH}} % GARCH

% Blockchain/Crypto
\newcommand{\Hash}{\mathrm{Hash}} % Hash function
\newcommand{\Block}{\mathcal{B}} % Block
\newcommand{\Chain}{\mathcal{C}} % Chain

% Real numbers, integers
\newcommand{\R}{\mathbb{R}}
\newcommand{\Z}{\mathbb{Z}}
\newcommand{\N}{\mathbb{N}}

% ======================= TIKZ STYLES =======================
% Styles for finance-related diagrams
\tikzstyle{process} = [rectangle, minimum width=3cm, minimum height=1cm, text centered, draw=dfblue, fill=dflightblue4, thick]
\tikzstyle{decision} = [diamond, minimum width=3cm, minimum height=1cm, text centered, draw=dfteal, fill=dflightblue4, thick]
\tikzstyle{arrow} = [thick,->,>=stealth,color=dfblue]
\tikzstyle{blockchain} = [rectangle, rounded corners, minimum width=2.5cm, minimum height=1cm, text centered, draw=dfteal, fill=dflightblue3, thick]
\tikzstyle{transaction} = [circle, minimum size=0.8cm, text centered, draw=dforange, fill=dflightblue4, thick]

% ======================= FOOTER TEMPLATE =======================
\setbeamertemplate{footline}{
    \hbox{\begin{beamercolorbox}[wd=\paperwidth,ht=2.5ex,dp=1ex,leftskip=.5em,rightskip=.5em]{author in head/foot}
    \tiny
    \textbf{Digital Finance} \hfill
    Joerg Osterrieder \hfill
    \insertdate \hfill
    Page \insertframenumber{} / \inserttotalframenumber
    \end{beamercolorbox}}
}

% ======================= SECTION DIVIDER TEMPLATE =======================
\AtBeginSection[]{
\begin{frame}[plain]
\vfill
\centering
\begin{beamercolorbox}[sep=12pt,center]{title}
\usebeamerfont{title}\LARGE\insertsection\par
\end{beamercolorbox}
\vfill
\end{frame}
}


% Additional TikZ libraries
\usetikzlibrary{calc,decorations.pathreplacing}

% ======================= DOCUMENT INFO =======================
\title[Smart Contracts Intro]{Smart Contracts}
\subtitle{Code as Agreement --- A 10-Slide Introduction}
\author{Joerg Osterrieder}
\institute{Digital Finance}
\date{}

\begin{document}

% =======================================================================
% SLIDE 1: WHY --- TikZ Comic (Dilemma, dfteal-dominant)
% =======================================================================
\begin{frame}[t]{What If a Contract Could Enforce Itself --- With No Lawyer, No Court, No Appeal?}
\begin{columns}[T]
\begin{column}{0.55\textwidth}
\small
Two parties make a deal. One side must trust the other to follow through. Lawyers cost money, courts take time, enforcement is uncertain. What if the agreement itself could guarantee compliance?

\vspace{2mm}
\footnotesize
\textbf{Three problems with traditional enforcement:}
\begin{enumerate}\compactlist
\item \textbf{Cost} --- intermediaries extract fees at every step
\item \textbf{Speed} --- resolution takes weeks to years
\item \textbf{Access} --- not everyone can afford legal representation
\end{enumerate}

\vspace{2mm}
\footnotesize
These barriers mean that for small transactions, enforcement is not worth the cost. For large ones, it favors whoever has more resources.
\end{column}
\begin{column}{0.42\textwidth}
\begin{center}
\begin{tikzpicture}[scale=0.75]
% --- Table ---
\draw[thick, mlgray, fill=mllavender4, rounded corners=2pt] (0.3,2.0) rectangle (3.7,2.4);

% --- Left figure (seller) ---
\draw[thick, dfteal] (1.0,3.4) circle (0.15); % head
\draw[thick, dfteal] (1.0,3.25) -- (1.0,2.7); % body
\draw[thick, dfteal] (1.0,3.0) -- (0.7,2.8); % left arm
\draw[thick, dfteal] (1.0,3.0) -- (1.3,2.8); % right arm
\draw[thick, dfteal] (1.0,2.7) -- (0.8,2.4); % left leg
\draw[thick, dfteal] (1.0,2.7) -- (1.2,2.4); % right leg

% --- Right figure (buyer) ---
\draw[thick, dfteal] (3.0,3.4) circle (0.15); % head
\draw[thick, dfteal] (3.0,3.25) -- (3.0,2.7); % body
\draw[thick, dfteal] (3.0,3.0) -- (2.7,2.8); % left arm
\draw[thick, dfteal] (3.0,3.0) -- (3.3,2.8); % right arm
\draw[thick, dfteal] (3.0,2.7) -- (2.8,2.4); % left leg
\draw[thick, dfteal] (3.0,2.7) -- (3.2,2.4); % right leg

% --- Contract on table ---
\draw[thick, mlgray, fill=white] (1.7,2.15) rectangle (2.3,2.55);
\node[font=\tiny, mlgray] at (2.0,2.35) {Deal};

% --- Thought bubble left: "Will they pay?" ---
\draw[thick, mlgray, rounded corners=2pt] (-0.4,3.7) rectangle (1.1,4.1);
\node[font=\tiny] at (0.35,3.9) {Will they pay?};
\fill[mlgray] (0.85,3.6) circle (0.04);
\fill[mlgray] (0.9,3.65) circle (0.03);

% --- Thought bubble right: "Will they deliver?" ---
\draw[thick, mlgray, rounded corners=2pt] (2.6,3.7) rectangle (4.3,4.1);
\node[font=\tiny] at (3.45,3.9) {Will they deliver?};
\fill[mlgray] (3.05,3.6) circle (0.04);
\fill[mlgray] (3.0,3.65) circle (0.03);

% --- Glowing smart contract box below ---
\draw[thick, dfteal, fill=dfteal!15, rounded corners=3pt] (0.8,0.5) rectangle (3.2,1.3);
\node[font=\tiny\bfseries, dfteal] at (2.0,1.05) {Smart Contract};
\node[font=\tiny, dfteal] at (2.0,0.75) {Self-enforcing code};

% --- Arrow from box up to table ---
\draw[->, ultra thick, dfteal] (2.0,1.3) -- (2.0,2.0);

% --- Bottom punchline ---
\node[font=\tiny\bfseries, mlpurple] at (2.0,0.15) {No judge. No delay. No escape.};
\end{tikzpicture}
\end{center}
\end{column}
\end{columns}

\vspace{1mm}
\begin{block}{}
\footnotesize A smart contract replaces the enforcer with code --- the agreement executes automatically when conditions are met, removing the need for trust in any third party.
\end{block}
\end{frame}

% =======================================================================
% SLIDE 2: FEEL --- Text-Only Prompt (Full-width)
% =======================================================================
\begin{frame}[t]{Think About a Promise Someone Broke --- What Would Change If a Machine Enforced It?}
\small
A friend borrows money and promises to pay you back by Friday. Friday comes. Nothing happens. You ask. They say next week. Next week becomes next month.

\vspace{3mm}
\small
Now imagine a different version:

\vspace{2mm}
\footnotesize
\begin{description}\compactlist
\item[\scriptsize The promise:] Your friend deposits the amount into a shared digital lockbox at the moment of borrowing.
\item[\scriptsize The rule:] If they do not send the payment by Friday, the lockbox automatically returns your original deposit.
\item[\scriptsize The guarantee:] Neither of you can change the rules after agreeing.
\end{description}

\vspace{3mm}
\small
No awkward conversation. No broken friendship. No court. The mechanism handled it.

\vspace{3mm}
\begin{exampleblock}{The Core Idea}
\footnotesize This is the core idea behind smart contracts: agreements where the enforcement mechanism is built into the agreement itself. The code does not forget, delay, or make exceptions.
\end{exampleblock}
\end{frame}

% =======================================================================
% SLIDE 3: WHAT --- Comparison Table (adjustbox)
% =======================================================================
\begin{frame}[t]{What Makes a Smart Contract Different from a Regular One?}
\begin{columns}[T]
\begin{column}{0.55\textwidth}
\begin{center}
\begin{adjustbox}{max width=\linewidth}
\scriptsize
\begin{tabular}{l c c}
\toprule
\textbf{Aspect} & \textbf{Traditional Contract} & \textbf{Smart Contract} \\
\midrule
Written in & Natural language & Programming code \\
Enforced by & Courts, lawyers & Automatic execution \\
Ambiguity & Possible, needs interpretation & Impossible, deterministic \\
Amendment & Renegotiation & New deployment required \\
Transparency & Private document & Public, auditable \\
\bottomrule
\end{tabular}
\end{adjustbox}
\end{center}

\vspace{2mm}
\footnotesize
The shift from language to code eliminates ambiguity but also eliminates flexibility --- you cannot plead extenuating circumstances to a program. What the code says is what the code does, regardless of intent or context.
\end{column}
\begin{column}{0.42\textwidth}
\small
\textbf{Key properties that distinguish smart contracts:}

\vspace{2mm}
\footnotesize
\begin{itemize}\compactlist
\item \textbf{Deterministic} --- same input always produces same output
\item \textbf{Immutable} --- once deployed, code cannot be changed
\item \textbf{Transparent} --- anyone can read and verify the code
\item \textbf{Self-executing} --- no intermediary triggers execution
\end{itemize}

\vspace{2mm}
\scriptsize
These properties taken together create a fundamentally new kind of agreement: one where compliance is not optional, and enforcement costs nothing.
\end{column}
\end{columns}

\vspace{1mm}
\begin{block}{}
\footnotesize The removal of ambiguity is both the greatest strength and the greatest risk. Code cannot exercise judgment --- it can only follow rules.
\end{block}
\end{frame}

% =======================================================================
% SLIDE 4: CASE --- Step Diagram (6 steps + 2 branches)
% =======================================================================
\begin{frame}[t]{Follow One Loan from Request to Repayment --- No Humans Involved}
\begin{columns}[T]
\begin{column}{0.55\textwidth}
\begin{center}
\begin{tikzpicture}[scale=0.65,
    stepbox/.style={rectangle, rounded corners=3pt, minimum width=2.8cm, minimum height=0.5cm, text centered, thick, font=\tiny},
    rejectbox/.style={rectangle, rounded corners=3pt, minimum width=1.6cm, minimum height=0.4cm, text centered, thick, font=\tiny}
]
% Step 1
\node[stepbox, draw=mlpurple, fill=mllavender4] (s1) at (0,0) {1. Deposit collateral};
% Step 2
\node[stepbox, draw=mlpurple, fill=mllavender3] (s2) at (0,-1.0) {2. Verify ratio};
% Step 3
\node[stepbox, draw=dfteal, fill=dfteal!20] (s3) at (0,-2.0) {3. Release loan};
% Step 4
\node[stepbox, draw=mlpurple, fill=mllavender4] (s4) at (0,-3.0) {4. Accrue interest};
% Step 5
\node[stepbox, draw=mlpurple, fill=mllavender3] (s5) at (0,-4.0) {5. Repay loan};
% Step 6
\node[stepbox, draw=dfgreen, fill=dfgreen!20] (s6) at (0,-5.0) {6. Return collateral};

% Arrows between steps
\draw[->, thick, mlpurple] (s1) -- (s2);
\draw[->, thick, mlpurple] (s2) -- (s3);
\draw[->, thick, mlpurple] (s3) -- (s4);
\draw[->, thick, mlpurple] (s4) -- (s5);
\draw[->, thick, mlpurple] (s5) -- (s6);

% Branch from step 2: rejection
\node[rejectbox, draw=dfred, fill=dfred!20] (r1) at (3.2,-1.0) {Loan rejected};
\draw[->, thick, dfred] (s2.east) -- (r1.west) node[midway, above, font=\tiny, dfred] {ratio low};

% Branch from step 4: liquidation
\node[rejectbox, draw=dfred, fill=dfred!20] (r2) at (3.2,-3.0) {Liquidation};
\draw[->, thick, dfred] (s4.east) -- (r2.west) node[midway, above, font=\tiny, dfred] {price drops};

% Numbered circles
\foreach \i/\y in {1/0, 2/-1.0, 3/-2.0, 4/-3.0, 5/-4.0, 6/-5.0} {
    \node[circle, fill=mlpurple, text=white, font=\tiny, inner sep=1pt, minimum size=0.3cm] at (-1.8,\y) {\i};
}
\end{tikzpicture}
\end{center}
\end{column}
\begin{column}{0.42\textwidth}
\small
\textbf{What happened in those six steps:}

\vspace{2mm}
\footnotesize
\begin{itemize}\compactlist
\item No loan officer reviewed the application
\item No credit score was checked --- only collateral value
\item No bank held the funds --- the contract did
\item No collection agency was needed --- the code liquidated automatically
\end{itemize}

\vspace{2mm}
\footnotesize
The entire process --- from application to repayment --- ran as code. Every decision point had a clear rule and an automatic outcome.
\end{column}
\end{columns}

\vspace{1mm}
\begin{block}{}
\footnotesize Every step that removes a human also removes a point of judgment. The loan is faster, cheaper, and more predictable --- but it cannot consider context, hardship, or exceptional circumstances.
\end{block}
\end{frame}

% =======================================================================
% SLIDE 5: HOW --- Side-by-Side Architecture
% =======================================================================
\begin{frame}[t]{Where Does a Smart Contract Actually Run --- And Who Pays for It?}
\begin{columns}[T]
\begin{column}{0.55\textwidth}
\begin{center}
\begin{tikzpicture}[scale=0.75,
    archbox/.style={rectangle, rounded corners=2pt, minimum width=2.0cm, minimum height=0.5cm, text centered, thick, font=\tiny}
]
% ===== LEFT STACK: Traditional =====
\node[font=\tiny\bfseries, mlpurple] at (1.2,4.5) {Traditional};

% User (stick figure)
\draw[thick, mlgray] (1.2,4.1) circle (0.12);
\draw[thick, mlgray] (1.2,3.98) -- (1.2,3.6);
\node[font=\tiny, mlgray] at (1.2,4.3) {User};

% Bank server
\node[archbox, draw=mlgray, fill=mlgray!15] (bank) at (1.2,3.0) {Bank Server};
\draw[->, thick, mlgray] (1.2,3.55) -- (1.2,3.25);

% Database (cylinder-like)
\draw[thick, mlgray, fill=mlgray!15] (0.5,1.8) rectangle (1.9,2.3);
\draw[thick, mlgray, fill=mlgray!15] (0.5,2.3) ellipse (0.7 and 0.15);
\node[font=\tiny, mlgray] at (1.2,2.0) {Database};
\draw[->, thick, mlgray] (1.2,2.75) -- (1.2,2.45);

% Label
\node[font=\tiny, mlgray, align=center] at (1.2,1.3) {\textit{One company}\\\textit{controls everything}};

% ===== DASHED SEPARATOR =====
\draw[dashed, thick, mlgray] (2.7,1.0) -- (2.7,4.6);

% ===== RIGHT STACK: Smart Contract =====
\node[font=\tiny\bfseries, mlpurple] at (4.3,4.5) {Smart Contract};

% User (stick figure)
\draw[thick, dfteal] (4.3,4.1) circle (0.12);
\draw[thick, dfteal] (4.3,3.98) -- (4.3,3.6);
\node[font=\tiny, dfteal] at (4.3,4.3) {User};

% Transaction
\node[archbox, draw=mlpurple, fill=mllavender4] (tx) at (4.3,3.0) {Transaction};
\draw[->, thick, dfteal] (4.3,3.55) -- (4.3,3.25);

% Virtual machine
\node[archbox, draw=mlpurple, fill=mlpurple!20, minimum width=2.2cm] (vm) at (4.3,2.2) {Virtual Machine};
\draw[->, thick, mlpurple] (4.3,2.75) -- (4.3,2.45);
\node[font=\tiny, mlpurple, align=center] at (6.0,2.2) {\textit{runs on}\\\textit{every node}};

% Blockchain state
\node[archbox, draw=dfgreen, fill=dfgreen!20] (state) at (4.3,1.4) {Blockchain State};
\draw[->, thick, mlpurple] (4.3,1.95) -- (4.3,1.65);

% Label
\node[font=\tiny, dfteal, align=center] at (4.3,0.7) {\textit{Thousands of nodes}\\\textit{agree on result}};

% Bottom comparison label
\node[font=\tiny\bfseries, mlpurple] at (2.7,0.2) {Single server vs.\ Global consensus};
\end{tikzpicture}
\end{center}
\end{column}
\begin{column}{0.42\textwidth}
\small
\textbf{Who pays for execution?}

\vspace{2mm}
\footnotesize
\begin{description}\compactlist
\item[\scriptsize Gas:] a fee measured in computation steps --- like paying postage by weight
\item[\scriptsize Simple operations:] cost very little, such as adding two numbers
\item[\scriptsize Storage writes:] cost thousands of times more, because data is saved permanently
\item[\scriptsize Failed transactions:] still cost gas --- you pay for the attempt even if it fails
\end{description}

\vspace{2mm}
\scriptsize
The fee structure incentivizes efficient code and prevents spam --- no one can run an infinite loop because they would run out of gas.
\end{column}
\end{columns}

\vspace{1mm}
\begin{block}{}
\footnotesize A smart contract trades cheap centralized execution for expensive decentralized consensus. The cost buys something specific: no single party can alter, censor, or shut down the program.
\end{block}
\end{frame}

% =======================================================================
% SLIDE 6: RISK --- TikZ Failure Comic (dfred-dominant)
% =======================================================================
\begin{frame}[t]{The Code Worked Perfectly --- So Why Did Everyone Lose Their Money?}
\begin{columns}[T]
\begin{column}{0.55\textwidth}
\small
A community pooled funds into a shared investment contract. The code was public. The rules were clear. Then someone found a loophole.

\vspace{2mm}
\footnotesize
\textbf{The attack pattern --- reentrancy:}
\begin{enumerate}\compactlist
\item Attacker calls the ``withdraw'' function
\item Contract sends funds to attacker
\item Before updating the balance, attacker's code calls ``withdraw'' again
\item Contract still shows the old balance --- sends funds again
\item Repeat until drained
\end{enumerate}

\vspace{2mm}
\footnotesize
The code executed exactly as written. Every step was valid according to the rules. The problem was not a bug in execution --- it was a flaw in logic.
\end{column}
\begin{column}{0.42\textwidth}
\begin{center}
\begin{tikzpicture}[scale=0.75]
% --- Vault ---
\draw[ultra thick, dfred, fill=dfred!8, rounded corners=3pt] (0.3,2.0) rectangle (3.7,4.2);
\node[font=\tiny\bfseries, dfred] at (2.0,3.95) {Community Vault};

% Money symbols inside vault
\node[font=\scriptsize, dfteal] at (1.0,3.4) {\$};
\node[font=\scriptsize, dfteal] at (1.5,3.0) {\$};
\node[font=\scriptsize, dfteal] at (2.5,3.4) {\$};
\node[font=\scriptsize, dfteal] at (3.0,3.0) {\$};
\node[font=\scriptsize, dfteal] at (2.0,3.2) {\$};

% Speech bubble from vault
\draw[thick, mlgray, rounded corners=2pt] (3.8,3.5) rectangle (5.6,4.1);
\node[font=\tiny, align=center] at (4.7,3.9) {Balance: OK!};
\node[font=\tiny, align=center] at (4.7,3.65) {Sending funds...};
\draw[thick, mlgray] (3.7,3.8) -- (3.8,3.75);

% --- Attacker figure (right side) ---
\draw[thick, dfred] (4.5,3.0) circle (0.15); % head
\draw[thick, dfred] (4.5,2.85) -- (4.5,2.3); % body
\draw[thick, dfred] (4.5,2.65) -- (4.2,2.4); % left arm
\draw[thick, dfred] (4.5,2.65) -- (4.8,2.4); % right arm
\draw[thick, dfred] (4.5,2.3) -- (4.3,2.0); % left leg
\draw[thick, dfred] (4.5,2.3) -- (4.7,2.0); % right leg

% Loop arrow from attacker back to vault
\draw[->, ultra thick, dfred, rounded corners=5pt] (4.2,2.5) -- (3.7,2.5) -- (3.7,3.2) -- (4.0,3.2);
\node[font=\tiny, dfred, rotate=90] at (3.9,2.85) {again!};

% Thought bubble attacker
\draw[thick, mlgray, rounded corners=2pt] (4.8,1.6) rectangle (6.0,2.0);
\node[font=\tiny] at (5.4,1.8) {Ask again...};
\fill[mlgray] (4.7,2.05) circle (0.03);

% --- Money flowing out ---
\draw[thick, dfred, fill=dfred!20] (1.5,1.5) rectangle (1.8,1.7);
\node[font=\tiny, dfred] at (1.65,1.6) {\$};
\draw[thick, dfred, fill=dfred!20] (2.0,1.2) rectangle (2.3,1.4);
\node[font=\tiny, dfred] at (2.15,1.3) {\$};
\draw[thick, dfred, fill=dfred!20] (2.5,0.9) rectangle (2.8,1.1);
\node[font=\tiny, dfred] at (2.65,1.0) {\$};

% --- Crowd below ---
\draw[thick, dfteal] (0.5,0.5) circle (0.08);
\draw[thick, dfteal] (0.9,0.5) circle (0.08);
\draw[thick, dfteal] (1.3,0.5) circle (0.08);
\draw[thick, dfteal] (1.7,0.5) circle (0.08);

% Crowd speech bubble
\draw[thick, mlgray, rounded corners=2pt] (0.2,0.0) rectangle (2.2,0.3);
\node[font=\tiny] at (1.2,0.15) {Where did it go?};
\draw[thick, mlgray] (1.1,0.3) -- (1.1,0.42);

% --- Bottom label ---
\node[font=\tiny\bfseries, dfred] at (2.5,-0.2) {The code did exactly what it was told.};
\end{tikzpicture}
\end{center}
\end{column}
\end{columns}

\vspace{1mm}
\begin{block}{}
\footnotesize Immutability is a double-edged sword. It guarantees that no one can tamper with the rules --- but it also guarantees that a flaw in those rules cannot be fixed after deployment.
\end{block}
\end{frame}

% =======================================================================
% SLIDE 7: WHERE --- pgfplots Stacked Bar Chart
% =======================================================================
\begin{frame}[t]{How Much Value Is Already Locked in Smart Contracts?}
\begin{columns}[T]
\begin{column}{0.55\textwidth}
\begin{center}
\begin{adjustbox}{max width=\linewidth}
\begin{tikzpicture}
\begin{axis}[
    width=7cm, height=4.5cm,
    ybar stacked,
    bar width=14pt,
    symbolic x coords={Lending,Trading,Staking,Bridges,Other},
    xtick=data,
    x tick label style={font=\tiny},
    ylabel={\scriptsize Share of total value (\%)},
    y tick label style={font=\tiny},
    ymin=0, ymax=105,
    ytick={0,25,50,75,100},
    legend style={at={(1.02,1.0)}, anchor=north west, font=\tiny},
    legend cell align={left},
    grid=none,
    enlarge x limits=0.15,
    area style
]
% Largest protocols
\addplot[fill=mlpurple, draw=mlpurple!80!black] coordinates
    {(Lending,45) (Trading,35) (Staking,55) (Bridges,40) (Other,30)};
% Mid-tier protocols
\addplot[fill=dfteal!60, draw=dfteal] coordinates
    {(Lending,30) (Trading,35) (Staking,25) (Bridges,30) (Other,35)};
% Long tail
\addplot[fill=mlorange!60, draw=mlorange] coordinates
    {(Lending,25) (Trading,30) (Staking,20) (Bridges,30) (Other,35)};
\legend{Largest protocols, Mid-tier protocols, Long tail}
\end{axis}
\end{tikzpicture}
\end{adjustbox}
\end{center}

\vspace{1mm}
\tiny\textit{Illustrative distribution based on public DeFi data patterns. Not actual protocol data.}
\end{column}
\begin{column}{0.42\textwidth}
\small
\textbf{What these contracts actually do:}

\vspace{2mm}
\footnotesize
\begin{description}\compactlist
\item[\scriptsize Lending:] Deposit collateral, borrow assets, earn interest --- all without a bank
\item[\scriptsize Trading:] Swap tokens directly through automated market makers --- no broker
\item[\scriptsize Staking:] Lock tokens to secure a network and earn rewards
\item[\scriptsize Bridges:] Move assets between different blockchains
\end{description}

\vspace{2mm}
\scriptsize
The concentration pattern matters: in most categories, a small number of protocols hold the majority of value. Smart contracts create winner-take-most dynamics.
\end{column}
\end{columns}

\vspace{1mm}
\begin{block}{}
\footnotesize Billions in value are already managed by code, not institutions. The question is no longer whether smart contracts work --- it is whether they work safely enough at this scale.
\end{block}
\end{frame}

% =======================================================================
% SLIDE 8: IMPACT --- Stakeholder Map (5 actors)
% =======================================================================
\begin{frame}[t]{Who Benefits from Unstoppable Code --- And Who Gets Hurt?}
\begin{columns}[T]
\begin{column}{0.55\textwidth}
\begin{center}
\begin{tikzpicture}[scale=0.75,
    actor/.style={rectangle, rounded corners=3pt, minimum width=1.6cm, minimum height=0.5cm, text centered, thick, font=\tiny, draw=mlgray, fill=mllavender4}
]
% Central node
\node[rectangle, rounded corners=4pt, minimum width=2.0cm, minimum height=0.6cm, text centered, thick, draw=mlpurple, fill=mlpurple!80, font=\tiny\bfseries, text=white] (center) at (2.5,2.5) {Smart Contracts};

% Actor 1: Top-left --- Borrowers without bank access (benefit)
\node[actor] (a1) at (0.2,4.5) {\shortstack{Unbanked\\borrowers}};
\draw[->, thick, dfgreen] (center) -- (a1) node[midway, left, font=\tiny, dfgreen] {access};

% Actor 2: Top-right --- Developers / Auditors (benefit)
\node[actor] (a2) at (4.8,4.5) {\shortstack{Developers \&\\auditors}};
\draw[->, thick, dfgreen] (center) -- (a2) node[midway, right, font=\tiny, dfgreen] {demand};

% Actor 3: Right --- Traditional intermediaries (harm)
\node[actor] (a3) at (5.5,2.0) {\shortstack{Traditional\\intermediaries}};
\draw[->, thick, dfred] (center) -- (a3) node[midway, below, font=\tiny, dfred] {displace};

% Actor 4: Bottom-right --- Regulators (ambiguous)
\node[actor] (a4) at (4.5,0.3) {\shortstack{Regulators}};
\draw[->, thick, mlorange] (center) -- (a4) node[midway, right, font=\tiny, mlorange] {challenge};

% Actor 5: Bottom-left --- Victims of exploits (harm)
\node[actor] (a5) at (0.5,0.3) {\shortstack{Exploit\\victims}};
\draw[->, thick, dfred] (center) -- (a5) node[midway, left, font=\tiny, dfred] {harm};

% Legend
\node[font=\tiny, dfgreen] at (1.0,-0.4) {Green = benefit};
\node[font=\tiny, dfred] at (2.5,-0.4) {Red = harm};
\node[font=\tiny, mlorange] at (4.0,-0.4) {Orange = ambiguous};
\end{tikzpicture}
\end{center}
\end{column}
\begin{column}{0.42\textwidth}
\small
\textbf{The distribution of impact is uneven:}

\vspace{2mm}
\footnotesize
\begin{itemize}\compactlist
\item \textbf{Winners:} Those previously excluded from financial services gain access. Developers and auditors face growing demand for their skills.
\item \textbf{Losers:} Traditional intermediaries lose revenue streams. Exploit victims have no recourse --- no insurance, no refund, no appeal.
\item \textbf{Ambiguous:} Regulators face a technology that operates across borders, resists censorship, and moves faster than legislation.
\end{itemize}

\vspace{2mm}
\scriptsize
The same properties that empower the unbanked also empower the attacker.
\end{column}
\end{columns}

\vspace{1mm}
\begin{block}{}
\footnotesize Unstoppable code is neutral. It serves whoever invokes it. Whether that is a net positive depends on the balance between new access created and new harm enabled.
\end{block}
\end{frame}

% =======================================================================
% SLIDE 9: SO WHAT --- Balance Scale
% =======================================================================
\begin{frame}[t]{Three Questions That Reveal Whether a Smart Contract Is Safe}
\begin{columns}[T]
\begin{column}{0.55\textwidth}
\small
Before trusting value to any smart contract, ask three questions. The answers will not tell you what to do --- but they will tell you what you are trusting.

\vspace{2mm}
\footnotesize
\begin{enumerate}\compactlist
\item \textbf{Has the code been audited --- and by whom?}\\
\scriptsize An audit reduces risk but does not eliminate it. Many exploited contracts had multiple audits. Ask who audited it, when, and what scope was covered.

\vspace{1.5mm}
\footnotesize
\item \textbf{Can the contract be upgraded --- and by whom?}\\
\scriptsize Upgradeable contracts can fix bugs, but they reintroduce the trust you were trying to remove. Whoever holds the upgrade key holds the power.

\vspace{1.5mm}
\footnotesize
\item \textbf{What happens if something goes wrong?}\\
\scriptsize Immutable contracts have no rollback. If there is no emergency pause, no insurance fund, and no governance process, then a flaw means total loss.
\end{enumerate}
\end{column}
\begin{column}{0.42\textwidth}
% Balance scale (modeled on cbdc_intro.tex slide 5 pattern)
\begin{center}
\begin{tikzpicture}[scale=0.75]
% Fulcrum triangle
\fill[mlgray!30] (2.0,1.5) -- (1.6,0.8) -- (2.4,0.8) -- cycle;
\draw[thick, mlgray] (2.0,1.5) -- (1.6,0.8) -- (2.4,0.8) -- cycle;

% Beam (slightly tilted to show tension)
\draw[ultra thick, mlpurple] (0.2,1.8) -- (3.8,1.2);

% Pivot
\fill[mlpurple] (2.0,1.5) circle (0.08);

% Left pan --- Automation + Access
\draw[thick, dfteal] (0.0,1.4) -- (0.4,1.4);
\draw[thick, dfteal] (0.0,1.4) -- (0.2,1.8);
\draw[thick, dfteal] (0.4,1.4) -- (0.2,1.8);
\node[font=\tiny\bfseries, dfteal, align=center] at (0.2,1.05) {Automation\\+ Access};

% Items on left pan
\node[font=\tiny, dfteal] at (0.2,0.65) {No gatekeepers};
\node[font=\tiny, dfteal] at (0.2,0.4) {Always available};
\node[font=\tiny, dfteal] at (0.2,0.15) {Transparent rules};

% Right pan --- Risk + Rigidity
\draw[thick, dfred] (3.6,0.8) -- (4.0,0.8);
\draw[thick, dfred] (3.6,0.8) -- (3.8,1.2);
\draw[thick, dfred] (4.0,0.8) -- (3.8,1.2);
\node[font=\tiny\bfseries, dfred, align=center] at (3.8,0.45) {Risk\\+ Rigidity};

% Items on right pan
\node[font=\tiny, dfred] at (3.8,0.05) {Bugs are permanent};
\node[font=\tiny, dfred] at (3.8,-0.2) {No appeals};
\node[font=\tiny, dfred] at (3.8,-0.45) {Code is final};

% Question mark at top center
\node[font=\small\bfseries, mlpurple] at (2.0,2.6) {?};

% Annotation arrows
\draw[->, thick, dfteal] (0.2,2.4) -- (0.2,2.0);
\node[font=\tiny, dfteal, align=center] at (0.2,2.7) {More\\access};
\draw[->, thick, dfred] (3.8,2.4) -- (3.8,1.6);
\node[font=\tiny, dfred, align=center] at (3.8,2.7) {More\\risk};

% Bottom annotation
\node[font=\tiny, mlgray] at (2.0,-0.7) {\textit{Every design choice tips the balance.}};
\end{tikzpicture}
\end{center}
\end{column}
\end{columns}

\vspace{1mm}
\begin{block}{}
\footnotesize Trustless does not mean risk-free. It means the risk has shifted from trusting people to trusting code. Whether that is an improvement depends on the quality of the code and the design of the system.
\end{block}
\end{frame}

% =======================================================================
% SLIDE 10: ACT --- Activity Frame (Full-width)
% =======================================================================
\begin{frame}[t]{Your Challenge: Audit This Contract Design}
\small
A startup proposes the following smart contract system for freelance payments:

\vspace{2mm}
\footnotesize
\begin{description}\compactlist
\item[\scriptsize The promise:] Clients deposit payment when hiring a freelancer. The contract releases funds when the work is marked complete.
\item[\scriptsize The rule:] The client marks work as complete, OR a deadline passes and the freelancer can claim the funds.
\item[\scriptsize The upgrade:] The startup holds an admin key that can pause the contract and redirect funds in case of disputes.
\end{description}

\vspace{3mm}
\small
Apply the three questions from the previous slide:

\vspace{2mm}
\footnotesize
\begin{enumerate}\compactlist
\item \textbf{Audit question:} What parts of this design could contain hidden risks? Where would you want an auditor to focus?
\item \textbf{Upgrade question:} The startup holds an admin key. What does this mean for the ``trustless'' claim? What could the startup do with this power?
\item \textbf{Failure question:} If the startup disappears, what happens to funds locked in the contract? Is there a recovery path?
\end{enumerate}

\vspace{3mm}
\begin{exampleblock}{No Single Right Answer}
\footnotesize There is no single right answer. The point is to practice evaluating trade-offs: the admin key solves the dispute problem but reintroduces centralized control. Every smart contract design involves this tension between safety and autonomy.
\end{exampleblock}
\end{frame}

\end{document}
